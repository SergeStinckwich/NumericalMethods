\ifx\wholebook\relax\else
\documentclass[twoside]{book}
\usepackage[active]{srcltx}
%\usepackage[LY1]{em}
\usepackage{epsfig}
\def\etc{{\it etc}}
\def\eg{{\it e.g.}}
\def\ie{{\it i.e.}}
\def\cf{{\it c.f.}\ }
\def\erf{\mathop{\rm erf}}
\def\sign{\mathop{\rm sign}}
\def\prob{\mathop{\rm Prob}}
\def\var{\mathop{\rm var}}
\def\mod{\mathop{\rm mod}}
\def\cor{\mathop{\rm cor}}
\def\cov{\mathop{\rm cov}}
\def\cl{\mathop{\rm CL}}
\def\kg{\mathop{\rm Kg}}
\def\patstyle#1{{\sc #1}}
\def\th{^{\mathop{\rm th}}}
\def\st#1{^{\mathop{\rm #1}}}
\def\note#1{\begin{quote}{\bf Note:} #1\end{quote}}
\def\braket#1{\left\langle #1\right\rangle}
\def\order#1{\let\o=#1{\cal O}\ifx\o 1$\left(n\right)$\else$\left(n^{#1}\right)$\fi}
\newtheorem{privListing}{Listing}[chapter]
\newenvironment{listing}{\vskip 3ex\hrule\vskip 1ex\begin{privListing}}{\end{privListing}\hrule\vskip 1ex}
\newtheorem{privExample}{Code example}[chapter]
\newenvironment{codeExample}{\begin{privExample}\begin{quote}\tt}{\end{quote}\end{privExample}}
\def\relboxl#1#2{\hbox to #1\hsize{#2\hfil}}
\def\relboxc#1#2{\hbox to #1\hsize{\hfil #2\hfil}}
\def\relboxr#1#2{\hbox to #1\hsize{\hfil #2}}
\def\transpose#1{{\bf #1}^{\mathop{\rm T}}}
\def\inverse#1{{\bf #1}^{-1}}
%\def\tm{$^{\mathop{\rm TM}}$}
\def\tm{ }
\newenvironment{mainEquation}{\marginpar[\vspace{3 ex} Main
equation$\Rightarrow$]{\vspace{3 ex}$\Leftarrow$Main
equation}\begin{equation}}{\end{equation}}
\def\rubrique#1{\paragraph{#1}\hfil\par\noindent}

\begin{document}
\fi

\begin{titlepage}
\hbox{\/}\vspace{12 ex} {\Huge \bf About this version}

\vspace{10 ex}
We would like to thank Didier Besset for his great book and for his gift of the source and implementation to the community. 


This version is a version without the Java implementation and Java reference. Our goal is to make this book slimmer and easier to read. Check the full version available at http://xxx.yyy.com if you want to get both the Java and Smalltalk implementations.

The implementation presented in this book is part of the SciSmalltalk library available at
http://. 


This version and the full version are maintained by S. Ducasse and S. Stinckwich.
Remember that we are all Charly.

\vspace{5 ex}
28 January 2015.
\end{titlepage}


\begin{titlepage}
\hbox{\/}\vspace{12 ex} {\Huge \bf Preface}

\vspace{10 ex} \begin{flushright} {\sl Si je savais une chose
utile \`{a} ma nation qui f\^{u}t ruineuse \`{a} une autre,\\je ne
la proposerais pas \`{a} mon prince,\\parce que je suis homme
avant d'\^etre Fran\c cais,\\parce que je suis n\'ecessairement
homme,\\et que je ne suis Fran\c cais que par hasard.}\footnote{If I
knew some trade useful to my country, but which would ruin
another, I would not disclose it to my ruler, because I am a man
before being French, because I belong to mankind while I am French
only by a twist of fate.}\\ Charles de Montesquieux
\end{flushright}




\vspace{8 ex} When I first encountered object-oriented
programming I immediately became highly enthusiastic about it,
mainly because of my mathematical inclination. After all I learned
to use computers as a high-energy physicist. In mathematics, a
new, high order concept is always based on previously defined,
simpler, concepts. Once a property is demonstrated for a given
concept it can be applied to any new concept sharing the same
premises as the original one. In object-oriented language, this is
called reuse and inheritance. Thus, numerical algorithms using
mathematical concepts that can be mapped directly into objects.
\par
This book is intended to be read by object-oriented programmers
who need to implement numerical methods in their applications. The
algorithms exposed here are mostly fundamental numerical
algorithms with a few advanced ones. The purpose of the book is to
show that implementing these algorithms in an object-oriented
language is feasible and quite easily feasible. We expect readers
to be able to implement their own favorite numerical algorithm
after seeing the examples discussed in this book.
\par
The scope of the book is limited. It is not a Bible about
numerical algorithms. Such Bible-like books already exist and are
quoted throughout the chapters. Instead I wanted to illustrate how
mapping between mathematical concepts and computer objects. I have
limited the book to algorithms, which I have implemented and used
in real applications over 12 years of object-oriented programming.
Thus, the reader can be certain that the algorithms have been
tested in the field.
\par
Because the intent of the book is showing numerical methods to
object-oriented programmers the code presented in commented in
depth. Each algorithm is presented with the same organization.
First the necessary equation are introduced with short
explanation. This book is not one about mathematics so
explanations are kept to a minimum. Then the general
object-oriented architecture of the algorithm is presented.
Finally, this book intending to be a practical one, the code
implementation is exposed. First how to use it, for readers who
are just interested in using the algorithm. Then, the code
implementation is discussed and presented.


As far as possible each algorithm is presented with such example
of use. I did not want to build contrived examples. Instead I have
used examples personally encountered in my professional life. Some
people may think that some examples are coming from esoteric
domains. This is not so. Each example has been selected for its
generality. The reader should study each example regardless of the
field of application and concentrate on the universal aspects of
it.

\rubrique{Acknowledgements} The author wishes to express his
thanks to the many people with whom he had interactions about the
object-oriented approach --- Smalltalk and Java in particular ---
on the various electronic forums. One special person is Kent Beck
whose controversial statements raised hell and started spirited
discussions. I also thank Kent for showing me tricks about the
Refactoring Browser and {\sl eXtreme Programming}. I also would
like to thank Eric Clayberg for pulling me out of a ditch more
than once and for making such fantastic Smalltalk tools.

A special mention goes to Prof. Donald Knuth for being an
inspiration for me and many other programmers with his series of
books {\sl The Art of Computer Programming}, and for making this
wonderful typesetting program \TeX. This present book was typeset
with \TeX and \LaTeX.

Furthermore, I would like to give credit to a few people without
whom this present book would never have been published. First,
Joseph Pelrine who persuaded me that what I was doing was worth
sharing with the rest of the object-oriented community.

The author expresses his most sincere thanks to the reviewers who
toiled on the early manuscripts of this book. Without their
open-mindedness this book would never made it to a publisher.

Special thanks go to David N. Smith for triggering interesting
thoughts about random number generators and to Dr. William Leo for
checking the equations.

Finally my immense gratitude is due to Dr. St\'ephane Ducasse of the
University of Bern who checked the orthodoxy of the Smalltalk code
and who did a terrific job of rendering the early manuscript not
only readable but entertaining.

\vspace{5 ex}\relboxr{1}{Genolier, 11 April 2000}

\end{titlepage}
\ifx\wholebook\relax\else\end{document}\fi
