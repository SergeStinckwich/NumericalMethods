\ifx\wholebook\relax\else
\documentclass[twoside]{book}
\usepackage[active]{srcltx}
\usepackage[LY1]{em}
\usepackage{epsfig}
\def\etc{{\it etc}}
\def\eg{{\it e.g.}}
\def\ie{{\it i.e.}}
\def\cf{{\it c.f.}\ }
\def\erf{\mathop{\rm erf}}
\def\sign{\mathop{\rm sign}}
\def\prob{\mathop{\rm Prob}}
\def\var{\mathop{\rm var}}
\def\mod{\mathop{\rm mod}}
\def\cor{\mathop{\rm cor}}
\def\cov{\mathop{\rm cov}}
\def\cl{\mathop{\rm CL}}
\def\kg{\mathop{\rm Kg}}
\def\patstyle#1{{\sc #1}}
\def\th{^{\mathop{\rm th}}}
\def\st#1{^{\mathop{\rm #1}}}
\def\note#1{\begin{quote}{\bf Note:} #1\end{quote}}
\def\braket#1{\left\langle #1\right\rangle}
\def\order#1{\let\o=#1{\cal O}\ifx\o 1$\left(n\right)$\else$\left(n^{#1}\right)$\fi}
\newtheorem{privListing}{Listing}[chapter]
\newenvironment{listing}{\vskip 3ex\hrule\vskip 1ex\begin{privListing}}{\end{privListing}\hrule\vskip 1ex}
\newtheorem{privExample}{Code example}[chapter]
\newenvironment{codeExample}{\begin{privExample}\begin{quote}\tt}{\end{quote}\end{privExample}}
\def\relboxl#1#2{\hbox to #1\hsize{#2\hfil}}
\def\relboxc#1#2{\hbox to #1\hsize{\hfil #2\hfil}}
\def\relboxr#1#2{\hbox to #1\hsize{\hfil #2}}
\def\transpose#1{{\bf #1}^{\mathop{\rm T}}}
\def\inverse#1{{\bf #1}^{-1}}
%\def\tm{$^{\mathop{\rm TM}}$}
\def\tm{ }
\newenvironment{mainEquation}{\marginpar[\vspace{3 ex} Main
equation$\Rightarrow$]{\vspace{3 ex}$\Leftarrow$Main
equation}\begin{equation}}{\end{equation}}
\def\rubrique#1{\paragraph{#1}\hfil\par\noindent}

\begin{document}
\fi

\chapter{Optimization}
\label{ch:minimization} \vspace{1 ex}
\begin{flushright}
{\sl Cours vite au but, mais gare � la chute.}\footnote{Run fast
to the goal, but beware of the fall.}\\ Alexandre Soljenitsyne
\end{flushright}
\vspace{1 ex} An optimization problem is a numerical problem where
the solution is characterized by the largest or smallest value of
a numerical function depending on several parameters. Such
function is often called the goal function. Many kinds of problems
can be expressed into optimization, that is, finding the maximum
or the minimum of a goal function. This technique has been applied
to a wide variety of fields going from operation research to game
playing or artificial intelligence. In chapter \ref{ch:estimation}
for example, the solution of maximum likelihood or least square
fits was obtained by finding the maximum, respectively the minimum
of a function.

In fact generations of high energy physicists have used the
general purpose minimization program MINUIT\footnote{F.James, M.
Roos, {\sl MINUIT --- a system for function minimization and
analysis of the parameter errors and corrections}, Comput. Phys.
Commun., 10 (1975) 343-367.} written by Fred James\footnote{I take
this opportunity to thank Fred for the many useful discussions we
have had on the subject of minimization.} of CERN to perform least
square fits and maximum likelihood fits.  To achieve generality,
MINUIT uses several strategies to reach a minimum. In this chapter
we shall discuss a few techniques and conclude with a program
quite similar in spirit to MINUIT. Our version, however, will not
have all the features offered by MINUIT.

If the goal function can be expressed with an analytical form, the
problem of optimization can be reduced into calculating the
derivatives of the goal function respective to all parameters, a
tedious but manageable job. In most cases, however, the goal
function cannot always be expressed analytically.

The classes described in this chapter are different in Smalltalk
and in Java. Therefore we present two class diagrams: figure
\ref{fig:soptimizingclasses} shows the Smalltalk class diagram and
figure \ref{fig:joptimizingclasses} shows the Java class diagram.
The main reason for the difference is the strong typing imposed in
Java preventing the reuse of instance variables.
\begin{figure}
\center\psfig{figure=Figures/Optimizing.eps, width=11cm}
\caption{Smalltak classes used in optimization}
\label{fig:soptimizingclasses}
\end{figure}
\begin{figure}
\center\psfig{figure=Figures/OptimizingJ.eps, width=11cm}
\caption{Java classes used in optimization}
\label{fig:joptimizingclasses}
\end{figure}


\section{Introduction}
\label{sec:optimum} Let us state the problem is general term. Let
$f\left({\bf x}\right)$ be a function of a vector ${\bf x}$ of
dimension $n$. The $n$-dimensional space is called the search
space of the problem. Depending on the problem the space can be
continuous or not. In this section we shall assume that the space
is continuous.

If the function is derivable, the gradient of the function
respective to the vector ${\bf x}$ must vanish at the optimum.
Finding the optimum of the function can be replaced by the problem
of finding the vector ${\bf x}$ such that:
\begin{equation}
\label{eq:mincondition}
  {d f\left({\bf x}\right) \over d{\bf x}} = 0.
\end{equation}
Unfortunately, the above equation is not a necessary condition for
an optimum. It can be either a maximum. a minimum or a saddle
point, that is a point where the function has a minimum in one
projection and a maximum in another projection. Furthermore, the
function may have several optima. Figure \ref{fig:localabsobulte}
shows an example of a function having two minima.
\begin{figure}
\center\psfig{figure=Figures/MinimumAbsLoc.eps, width=11cm}
\caption{Local and absolute optima} \label{fig:localabsobulte}
\end{figure}
Some problems require to find the absolute optimum of the
function. Thus, one must verify that the solution of
\ref{eq:mincondition} corresponds indeed to an optimum with the
expected properties. The reader can already see at this point that
searching for an optimum in the general case is a very difficult
task.

\noindent All optimization algorithms can be classified in two
broad categories:
\begin{itemize}
  \item {\sl Greedy algorithms}: these algorithms are characterized by a
  local search in the most promising direction. They are usually efficient
  and quite good at finding local optima. Among greedy algorithms,
  one must distinguish those requiring the evaluation of the
  function's derivatives.
  \item {\sl Random based algorithms}: these algorithms are using
  a random approach. They are not efficient; however, they are good at
  finding absolute optima. Simulated annealing \cite{Press} and
  genetic algorithms\cite{BerLin} belong to this class.
\end{itemize}
The table \ref{tb:optimizingalgorithms} lists the properties of
the algorithms presented in this chapter.
\begin{table}[h]
  \centering
  \caption{Optimizing algorithms presented in this book}\label{tb:optimizingalgorithms}
\vspace{1 ex}\begin{tabular}{| l | c | c |} \hline
  \hfil{\bf Name} & {\bf Category} & {\bf Derivatives} \\ \hline
  Extended Newton & greedy & yes \\
  Powell's hill climbing & greedy & no \\
  Simplex & greedy & no \\
  Genetic algorithm & random based & no \\ \hline
\end{tabular}
\end{table}


\section{Extended Newton algorithms}
Extended Newton algorithms are using a generalized version of
Newton's zero finding algorithm. These algorithms assume that the
function is continuous and has only one optimum in the region
where the search is initiated.

Let us expand the function $f\left({\bf x}\right)$ around a point
${\bf x}^{\left(0\right)}$ near the solution. We have in
components:
\begin{equation}
  f\left({\bf x}\right) = f\left[{\bf x}^{\left(0\right)}\right] +\sum_j
  \left.{\partial f\left({\bf x}\right) \over \partial x_j}\right|_{{\bf x}={\bf
  x}^{\left(0\right)}}
  \left[x_j-x^{\left(0\right)}_j\right].
\end{equation}
Using the expansion above into equation \ref{eq:mincondition}
yields:
\begin{equation}
\sum_j
  \left.{\partial^2 f\left({\bf x}\right) \over \partial x_i\partial x_j}\right|_{{\bf x}={\bf
  x}^{\left(0\right)}}
  \left[x_j-x^{\left(0\right)}_j\right]
  + \left.{\partial f\left({\bf x}\right) \over \partial x_i}\right|_{{\bf x}={\bf
  x}^{\left(0\right)}} =0.
\end{equation}
This equation can be written as a system of linear equations of
the form
\begin{equation}
  {\bf M}\Delta = {\bf c},
\end{equation}
where $\Delta_j =x_j-x^{\left(0\right)}_j$. The components of the
matrix ${\bf M}$ --- called the Hessian matrix --- are given by:
\begin{equation}
  m_{ij} = \left.{\partial^2 f\left({\bf x}\right) \over \partial x_i\partial x_j}\right|_{{\bf x}={\bf
  x}^{\left(0\right)}},
\end{equation}
and the components of the vector ${\bf c}$ are given by:
\begin{equation}
  c_i = -\left.{\partial f\left({\bf x}\right) \over \partial x_i}\right|_{{\bf x}={\bf
  x}^{\left(0\right)}}.
\end{equation}
Like in section \ref{sec:lsfnonlin} one can iterate this process
by replacing ${\bf x}^{\left(0\right)}$ with ${\bf
x}^{\left(0\right)}+\Delta$. This process is actually equivalent
to the Newton zero finding method (\cf section \ref{sec:newton}).
The final solution is a minimum if the matrix ${\bf M}$ is
positive definite; else it is a maximum.

This technique is used by MINUIT in the vicinity of the goal
function's optimum. It is the region where the algorithm described
above works well. Far from the optimum, the risk of reaching a
point where the matrix ${\bf M}$ cannot be inverted is quite high
in general. In addition, the extended Newton algorithm requires
that the second order derivatives of the function can be computed
analytically; at least the first order derivatives must be
provided, otherwise the cost of computation at each step becomes
prohibitive. A concrete implementation of the technique is not
given here. The reader can find in this book all the necessary
tools to make such an implementation. It is left as a exercise for
the reader. In the rest of this chapter, we shall present other
methods which work without an analytical knowledge of the
function.

\section{Hill climbing algorithms}
Hill climbing is a generic term covering many algorithms trying to
reach an optimum by determining the optimum along successive
directions. The general algorithm is outlined below.
\begin{enumerate}
  \item select an initial point ${\bf x}_0$ and a direction ${\bf v}$;
  \item find ${\bf x}_1$, the optimum of the function along the selected
  direction;
  \item if convergence is attained, terminate the algorithm;
  \item set ${\bf x}_0={\bf x}_1$, select a different direction and go back to step 2.
\end{enumerate}
The simplest of these algorithms simply follows each axis in turn
until a convergence is reached. More elaborate algorithms
exist\cite{Press}. One of them is described in section
\ref{sec:powell}.

Hill climbing algorithms can be applied to any continuous
function, especially when the function's derivatives are not
easily calculated. The core of the hill climbing algorithm is
finding the optimum along one direction. Let ${\bf v}$ be the
direction, then finding the optimum of the vector function
$f\left({\bf x}\right)$ along the direction ${\bf v}$ starting
from point ${\bf x}_0$ is equivalent to finding the optimum of the
one-variable function $g\left(\lambda\right)=f\left({\bf
x}_0+\lambda{\bf v}\right)$.

Therefore, in order to implement a hill climbing algorithm, we
first need to implement an algorithm able to find the optimum of a
one-variable function. This is the topic of the sections
\ref{sec:optonedim} and \ref{sec:bracket}. Before this, we need to
discuss the implementation details providing a common framework to
all classes discussed in the rest of this chapter.

\subsection{Optimizing --- General implementation}
\label{sec:goptonedim} At this point the reader may be a little
puzzled by the use of {\sl optimum} instead of speaking of minimum
or maximum. We shall now disclose a general implementation which
works both for finding a minimum or a maximum. This should not
come to a surprise since, in mathematics, a minimum or a maximum
are both very similar --- position where the derivative of a
function vanishes
--- and can be easily turned into each other --- e.g. by negating the
function.

To implement a general purpose optimizing framework, we introduce
two new classes: {\tt MinimizingPoint} and {\tt MaximizingPoint},
 a subclass of {\tt MinimizingPoint}. These two classes
are used as \patstyle{Strategy} by the optimizing algorithms. The
class {\tt MinimizingPoint} has two instance variables
\begin{description}
  \item[\tt value] the value of the goal function, that is
  $g\left(\lambda\right)$ or $f\left({\bf x}\right)$;
  \item[\tt position] the position at which the function has been
  evaluated, that is $\lambda$ or ${\bf x}$.
\end{description}
The class {\tt MinimizingPoint} contains most of the methods. The
only method overloaded by the class  {\tt MaximizingPoint} is the
method {\tt betterThan}, which tells whether an optimizing point
is better than another. The method {\tt betterThan} can be used in
all parts of the optimizing algorithms to find out which point is
the optimum so far. In algorithms working in multiple dimensions,
the method {\tt betterThan} is also used to sort the points from
the best to the worst. In Java, the architecture is a little more
complex because of typing requirements, but the basic design
concept is the same.

A convenience instance creation method allows to create instances
for a given function with a given argument. The instance is then
initialized with the function's value evaluated at the argument.
Thus, all optimizing algorithms described here do not call the
goal function explicitly.

Otherwise the implementation of the one dimensional optimum search
uses the general framework of the iterative process. More
specifically it uses the class {\tt FunctionalIterator} described
in section \ref{sec:iterrel}.

A final remark concerns the method {\tt initializeIteration}. The
golden search algorithm assume that the 3 points $\lambda_0$,
$\lambda_1$ and $\lambda_2$ have been determined. What if they
have not been? In this case, the method {\tt initializeIteration}
uses the optimum bracket finder described in section
\ref{sec:bracket}

\subsection{Common optimizing classes --- Smalltalk implementation}
\label{sec:sgeneralOpt} \marginpar{Figure
\ref{fig:soptimizingclasses} with the boxes {\bf
FunctionOptimizer}, {\bf MinimizingPoint}, {\bf MaximizingPoint},
{\bf ProjectedOneVariableFunction} and {\bf
VectorProjectedFunction} grayed.} In Smalltalk we have two classes
of optimizing points: {\tt DhbMinimizingPoint} and its subclass
{\tt DhbMaximizingPoint}. These classes are shown in listing
\ref{ls:optimizerCommon}. The class {\tt DhbFunctionOptimizer} is
in charge of handling the management of the optimizing points.
This clas is shown in listing \ref{ls:optimizerAbstract}.

\noindent The class {\tt DhbMinimizingPoint} has the following
instance variables:
\begin{description}
  \item[\tt position] contains the position at which the function
  is evaluated; this instance variable is a number if the function
  to optimize is a one variable function and an array or a vector
  if the function to evaluate is a function of many variables;
  \item[\tt value] contains the value of the function evaluated at the point's position;
\end{description}
Accessor methods corresponding to these variables are supplied. As
we noted in section \ref{sec:goptonedim}, the only method
redefined by the subclass {\tt DhbMaximizingPoint} is the method
{\tt betterThan:} used to decide whether a point is better than
another.

Optimizing points are created with the convenience method {\tt
vector:function:} which evaluates the function supplied as second
argument at the position supplied as the first argument.
\begin{listing} Smalltalk classes common to all optimizing classes \label{ls:optimizerCommon}
$$\halign{ #\hfil&\quad#\hfil\cr {\sl Class}& {\Large\bf DhbMinimizingPoint}\cr
{\sl Subclass of }&{\tt Object}\cr\noalign{\vskip 1ex}

{\sl Instance variable names:}&\parbox[t]{4 in}{\tt  value position }\cr\noalign{\vskip 1ex}}$$


Class methods
{\parskip 1ex\par\noindent}
{\bf new:} {\tt aVector} {\bf value:} {\tt aNumber}
\begin{verbatim}
    ^self new vector: aVector; value: aNumber; yourself

\end{verbatim}
{\bf vector:} {\tt aVector} {\bf function:} {\tt aFunction}
\begin{verbatim}
    ^self new: aVector value: (aFunction value: aVector)

\end{verbatim}



Instance methods
{\parskip 1ex\par\noindent}
{\bf betterThan:} {\tt anOptimizingPoint}
\begin{verbatim}
    ^value < anOptimizingPoint value

\end{verbatim}
{\bf position}
\begin{verbatim}
    ^position

\end{verbatim}
{\bf printOn:} {\tt aStream}
\begin{verbatim}
    position printOn: aStream.
    aStream
        nextPut: $:;
        space.
    value printOn: aStream

\end{verbatim}
{\bf value}
\begin{verbatim}
    ^value

\end{verbatim}
{\bf value:} {\tt aNumber}
\begin{verbatim}
    value := aNumber.

\end{verbatim}
{\bf vector:} {\tt aVector}
\begin{verbatim}
    position := aVector

\end{verbatim}


$$\halign{ #\hfil&\quad#\hfil\cr {\sl Class}& {\Large\bf DhbMaximizingPoint}\cr
{\sl Subclass of }&{\tt DhbMinimizingPoint}\cr\noalign{\vskip 1ex}
}$$


Instance methods
{\parskip 1ex\par\noindent}
{\bf betterThan:} {\tt anOptimizingPoint}
\begin{verbatim}
    ^ value > anOptimizingPoint value
\end{verbatim}


\end{listing}

The class {\tt DhbFunctionOptimizer} is in charge of handling the
optimizing points. it has the following instance variables:
\begin{description}
  \item[\tt optimizingPointClass] is the class of the optimizing
  points used as \patstyle{Strategy} by the optimizer; it is used
  to create instances of points at a given position for a given
  function;
  \item[\tt bestPoints] contains a sorted collection of optimizing
  points; the best point is the first and the worst point is the
  last; all optimizers rely on the fact that sorting is done by this sorted collection.
\end{description}
The method {\tt addPointAt:} creates an optimizing point at the
position supplied as argument and adds this point to the
collection of best points. Since that collection is sorted, one is
always certain to find the best result in the first position. This
fact is used by the method {\tt finalizeIterations}, which
retrieves the result from the collection of best points.

Instances of the function optimizer are created with the two
convenience methods {\tt minimizingFuntion:} and {\tt
maximizingFuntion:} helping to define the type of optimum. An
additional convenience method, {\tt forOptimizer:} allows to
create a new optimizer with the same strategy --- that is, the
same class of optimizing points --- and the same function as the
optimizer supplied as argument. This method is used to create
optimizers used in intermediate steps.

Because finding an optimum cannot be determined numerically with
high precision \cite{Press} the class {\tt DhbFunctionOptimizer}
redefines the method {\tt defaultPrecision} to be 100 times the
default numerical precision.
\begin{listing} Smalltalk abstract class for all optimizing classes \label{ls:optimizerAbstract}
$$\halign{ #\hfil&\quad#\hfil\cr {\sl Class}& {\Large\bf DhbFunctionOptimizer}\cr
{\sl Subclass of }&{\tt DhbFunctionalIterator}\cr\noalign{\vskip 1ex}

{\sl Instance variable names:}&\parbox[t]{4 in}{\tt  optimizingPointClass bestPoints }\cr\noalign{\vskip 1ex}}$$


Class methods
{\parskip 1ex\par\noindent}
{\bf defaultPrecision}
\begin{verbatim}
    ^ super defaultPrecision * 100
\end{verbatim}
{\bf forOptimizer:} {\tt aFunctionOptimizer}
\begin{verbatim}
    ^ self new initializeForOptimizer: aFunctionOptimizer
\end{verbatim}
{\bf maximizingFunction:} {\tt aFunction}
\begin{verbatim}
    ^ super new initializeAsMaximizer; setFunction: aFunction
\end{verbatim}
{\bf minimizingFunction:} {\tt aFunction}
\begin{verbatim}
    ^ super new initializeAsMinimizer; setFunction: aFunction
\end{verbatim}



Instance methods
{\parskip 1ex\par\noindent}
{\bf addPointAt:} {\tt aNumber}
\begin{verbatim}
    bestPoints add: (optimizingPointClass vector: aNumber
                                        function: functionBlock)
\end{verbatim}
{\bf bestPoints}
\begin{verbatim}
    ^ bestPoints
\end{verbatim}
{\bf finalizeIterations}
\begin{verbatim}
    result := bestPoints first position.
\end{verbatim}
{\bf functionBlock}
\begin{verbatim}
    ^ functionBlock
\end{verbatim}
{\bf initialize}
\begin{verbatim}
    bestPoints := SortedCollection sortBlock:
                                         [ :a :b | a betterThan: b ].
    ^ super initialize
\end{verbatim}
{\bf initializeAsMaximizer}
\begin{verbatim}
    optimizingPointClass := DhbMaximizingPoint.
    ^ self initialize
\end{verbatim}
{\bf initializeAsMinimizer}
\begin{verbatim}
    optimizingPointClass := DhbMinimizingPoint.
    ^ self
\end{verbatim}
{\bf initializeForOptimizer:} {\tt aFunctionOptimizer}
\begin{verbatim}
    optimizingPointClass := aFunctionOptimizer pointClass.
    functionBlock := aFunctionOptimizer functionBlock.
    ^ self initialize
\end{verbatim}
{\bf initialValue:} {\tt aVector}
\begin{verbatim}
    result := aVector copy.
\end{verbatim}
{\bf pointClass}
\begin{verbatim}
    ^ optimizingPointClass
\end{verbatim}
{\bf printOn:} {\tt aStream}
\begin{verbatim}
    super printOn: aStream.
    bestPoints do: [ :each | aStream cr. each printOn: aStream ].
\end{verbatim}

\end{listing}

In order to find an optimum along a given direction, one needs to
construct an object able to transform a vector function into a one
variable function. The class {\tt DhbProjectedOneVariableFunction}
and its subclass {\tt DhbVectorProjectedFunction} provide this
functionality. They are shown in listing
\ref{ls:projectedfunctions}. The class {\tt
DhbProjectedOneVariableFunction} has the following instance
variables:
\begin{description}
  \item[\tt function] the goal function $f\left({\bf x}\right)$;
  \item[\tt argument] the vector argument of the goal function,
  that is the vector ${\bf x}$;
  \item[\tt index] the index of the axis along which the function
  is projected.
\end{description}
The instance variables {\tt argument} and {\tt index} can be read
and modified using direct accessor methods. The goal function is
set only at creation time: the instance creation method {\tt
function:} take the goal function as argument. A convenience
method {\tt bumpIndex} allows to alter the index in circular
fashion\footnote{We do not give the implementation of the simplest
of the hill climbing algorithms using alternatively each axes of
the reference system. This implementation, which uses the method
{\tt bumpIndex}, is left as an exercise for the reader.}.

The class {\tt DhbVectorProjectedFunction} has no additional
variables. Instead it is reusing the instance variable {\tt index}
as the direction along which the function is evaluated. For
clarity, the accessor methods have been renamed {\tt direction},
{\tt direction:}, {\tt origin} and {\tt origin:}.

For both classes, the method {\tt argumentAt:} returns the
argument vector for the goal function, that is the vector ${\bf
x}$. The method {\tt value:} returns the value of the function
$g\left(\lambda\right)$ for the supplied argument $\lambda$.

\begin{listing} Smalltalk projected function classes \label{ls:projectedfunctions}
$$\halign{ #\hfil&\quad#\hfil\cr {\sl Class}& {\Large\bf DhbProjectedOneVariableFunction}\cr
{\sl Subclass of }&{\tt Object}\cr\noalign{\vskip 1ex}

{\sl Instance variable names:}&\parbox[t]{4 in}{\tt  index function argument }\cr\noalign{\vskip 1ex}}$$


Class methods
{\parskip 1ex\par\noindent}
{\bf function:} {\tt aVectorFunction}
\begin{verbatim}
    ^ super new initialize: aVectorFunction
\end{verbatim}


Instance methods
{\parskip 1ex\par\noindent}
{\bf argumentWith:} {\tt aNumber}
\begin{verbatim}
    ^ argument at: index put: aNumber; yourself
\end{verbatim}
{\bf bumpIndex}
\begin{verbatim}
    index isNil
        ifTrue: [ index := 1]
        ifFalse: [ index := index + 1.
                  index > argument size
                    ifTrue: [ index := 1].
                ].
\end{verbatim}
{\bf index}
\begin{verbatim}
    index isNil
        ifTrue: [ index := 1].
    ^ index
\end{verbatim}
{\bf initialize:} {\tt aFunction}
\begin{verbatim}
    function := aFunction.
    ^ self
\end{verbatim}
{\bf setArgument:} {\tt anArrayOrVector}
\begin{verbatim}
    argument := anArrayOrVector copy.
\end{verbatim}
{\bf setIndex:} {\tt anInteger}
\begin{verbatim}
    index := anInteger.
\end{verbatim}
{\bf value:} {\tt aNumber}
\begin{verbatim}
    ^ function value: ( self argumentWith: aNumber)
\end{verbatim}


$$\halign{ #\hfil&\quad#\hfil\cr {\sl Class}& {\Large\bf DhbVectorProjectedFunction}\cr
{\sl Subclass of }&{\tt DhbProjectedOneVariableFunction}\cr\noalign{\vskip 1ex}
}$$


Instance methods
{\parskip 1ex\par\noindent}
{\bf argumentWith:} {\tt aNumber}
\begin{verbatim}
    ^ aNumber * self direction + self origin
\end{verbatim}
{\bf direction}
\begin{verbatim}
    ^ index
\end{verbatim}
{\bf direction:} {\tt aVector}
\begin{verbatim}
    index := aVector.
\end{verbatim}
{\bf origin}
\begin{verbatim}
    ^ argument
\end{verbatim}
{\bf origin:} {\tt aVector}
\begin{verbatim}
    argument := aVector.
\end{verbatim}
{\bf printOn:} {\tt aStream}
\begin{verbatim}
    self origin printOn: aStream.
    aStream nextPutAll: ' ('.
    self direction printOn: aStream.
    aStream nextPut: $).
\end{verbatim}


\end{listing}

\subsection{Common optimizing classes --- Java implementation}
\label{sec:jgeneralOpt} \marginpar{Figure
\ref{fig:joptimizingclasses} with the boxes {\bf OptimizingPoint},
{\bf MinimizingPoint}, {\bf MaximizingPoint}, {\bf
ManyVariableFunction}, {\bf OptimizingVector}, {\bf
MinimizingVector}, {\bf MaximizingVector}, {\bf
OptimizingFactory}, {\bf MinimizingFactory}, {\bf
MaximizingFactory} and {\bf MultiVariableOptimizer} grayed.} In
Java we have two distinct hierarchies of optimizing points (\cf
figure \ref{fig:joptimizingclasses}). The abstract class {\tt
OptimizingPoint} has two subclasses {\tt MinimizingPoint} and {\tt
MaximizingPoint}. These classes are shown in listing
\ref{lj:optimizerCommonPoint}. The abstract class {\tt
OptimizingPoint} has two instance variables:
\begin{description}
  \item[\tt position] a {\tt double} containing the position at which the function
  is evaluated;
  \item[\tt value] a {\tt double} containing the value of the function evaluated at the point's position;
\end{description}
Both instance variables have a corresponding getter accessor
method, but no setter method. Otherwise, the functional relation
between {\tt position} and {\tt value} could not be ensured. As we
noted in section \ref{sec:goptonedim}, the only method redefined
by the subclass {\tt MaximizingPoint} is the method {\tt
betterThan} used to decide whether a point is better than another.
Instances of optimizing points are created with a single
constructor method taking as arguments the position at which the
function must be evaluated and the function itself. The supplied
function must implement the {\tt OneVariableFunction} interface
described in section \ref{sec:jvFunction}.

\begin{listing} Java optimizing point classes \label{lj:optimizerCommonPoint}
\input{Java/DhbOptimizing/OptimizingPoint.java}
\input{Java/DhbOptimizing/MinimizingPoint.java}
\input{Java/DhbOptimizing/MaximizingPoint.java}
\end{listing}

The abstract class {\tt OptimizingVector} has two subclasses {\tt
MinimizingVector} and {\tt Maximizingvector}. The only differences
with the corresponding optimizing point classes are that the
instance variable {\tt position} is an array of {\tt double} and
that the function supplied as the second argument of the
constructor method must implement the {\tt ManyVariableFunction}
interface. These classes and the interface {\tt
ManyVariableFunction} are showns in listing
\ref{lj:optimizerCommonVector}. The need for the additional
hierarchy comes primarily from the fact that we cannot handle {\tt
double} and array of double as a common object since these
entities are primitive types and not objects.

\begin{listing} Java optimizing vector classes \label{lj:optimizerCommonVector}
\input{Java/DhbInterfaces/ManyVariableFunction.java}
\input{Java/DhbOptimizing/OptimizingVector.java}
\input{Java/DhbOptimizing/MinimizingVector.java}
\input{Java/DhbOptimizing/MaximizingVector.java}
\end{listing}

The abstract class {\tt OptimizingPointFactory} and its two
concrete subclasses, {\tt MinimizingPointFactory} and {\tt
MaximizingPointFactory}, are in charge of creating the optimizing
points. These factory classes\footnote{For people reading code in
both languages: Smalltalk does not need any factory: objects of
type {\tt Class} are instance factories built into the language.
Reflection in Java could have been used here to build an
architecture similar to that of Smalltalk. The use of reflection,
however, implies the use of casting, which, in my humble opinion,
spoils the purpose of strong typing.} play the role of the
optimizing \patstyle{Strategy} described in section
\ref{sec:goptonedim}. The concrete classes implement the two
methods {\tt createPoint} and {\tt createVector}. The method {\tt
createPoint} creates an optimizing point of the desired type for a
supplied position (a {\tt double}) and function (a {\tt
OneVariableFunction}). The method {\tt createVector} creates an
optimizing vector of the desired type for a supplied position (an
array of {\tt double}) and function (a {\tt
ManyVariableFunction}). A convenience method with the same name
takes as first argument a {\tt DhbVector} (\cf section
\ref{sec:jlinearalgebra}). Since the factory is able to create
either points or vectors it can be reused between optimizer
working with one or many variable functions.

\begin{listing} Java optimizing point factory classes \label{lj:optimizerCommon}
\input{Java/DhbOptimizing/OptimizingPointFactory.java}
\input{Java/DhbOptimizing/MinimizingPointFactory.java}
\input{Java/DhbOptimizing/MaximizingPointFactory.java}
\end{listing}

In the Smalltalk implementation we have been able to bring all
optimizing classes under a single abstract class. The Java
architecture is similar. In the case of Java, however, the
abstract class only handles optimization in $n$-dimensional space.

Listing \ref{lj:optimizerAbtractCommon} shows the code of class
{\tt MultiVariableOptimizer}. This class has the following
instance variables:
\begin{description}
  \item[\tt f] the goal function; this object must implement the
  {\tt ManyVariableFunction} interface shown in listing
  \ref{lj:optimizerCommonVector}.
  \item[\tt pointFactory] the factory used to create optimizing
  points; this must be one concrete instance of the classes shown
  in listing \ref{lj:optimizerCommon}.
  \item[\tt result] an array of double containing the initial value
  where to start the algorithm; at the end of the
  algorithm, this variable contains the position of the optimum if
  convergence was attained;
\end{description}
The default constructor method provided by the abstract class take
three arguments: the goal function, the optimizing point factory
and the initial value. These three arguments correspond to the
three instance variables of the class.

The method {\tt setInitialValue} allows to change the initial
value in case another search is made with the same instance. The
accessor method {\tt getResult} allows one to retrieve the result.

The method {\tt sortPoints} sorts a supplied array of optimizing
points according to their \patstyle{Strategy}. Since sorting is
made {\it in situ}, a bubble sort algorithm is used.

\begin{listing} Java abstract class for all optimizing classes \label{lj:optimizerAbtractCommon}
\input{Java/DhbOptimizing/MultiVariableOptimizer.java}
\end{listing}

In order to find an optimum along a given direction, one needs to
construct an object able to transform a vector function into a one
variable function. The class {\tt VectorProjectedFunction}
providing this functionality has the following instance variables:
\begin{description}
  \item[\tt f] the goal function $f\left({\bf x}\right)$;
  \item[\tt origin] the vector argument of the goal function,
  that is the vector ${\bf x}_0$;
  \item[\tt direction] the direction along which the function
  is projected,
  that is the vector ${\bf v}$.
\end{description}
The constructor method takes three argument corresponding to the
three instance variables. In addition, the instance variables {\tt
argument} and {\tt direction} can be read and modified using
direct accessor methods. For convenience the set methods exist in
two versions of the argument: an array of {\tt double} or a {\tt
DhbVector}. This is also true for the constructor method.

The method {\tt argumentAt:} returns the argument vector for the
goal function, that is the vector ${\bf x}$. The method {\tt
value:} returns the value of the function $g\left(\lambda\right)$
for the supplied argument $\lambda$.
\begin{listing} Java projected function class \label{lj:projectedfunctions}
\input{Java/DhbOptimizing/VectorProjectedFunction.java}
\end{listing}

\section{Optimizing in one dimension}
\label{sec:optonedim} To find the optimum of a one-variable
function, $g\left(\lambda\right)$, whose derivative is unknown,
the most robust algorithm is an algorithm similar to the bisection
algorithm described in section \ref{sec:bisection}.

Let us assume that we have found three points $\lambda_0$,
$\lambda_1$ and $\lambda_2$ such that
$\lambda_0<\lambda_1<\lambda_2$ and such that
$g\left(\lambda_1\right)$ is better than both
$g\left(\lambda_0\right)$ and $g\left(\lambda_2\right)$. If the
function $g$ is continuous over the interval
$\left[\lambda_0,\lambda_2\right]$, then we are certain that an
optimum of the function is located in the interval
$\left[\lambda_0,\lambda_2\right]$. As for the bisection
algorithm, we shall try to find a new triplet of values with
similar properties while reducing the size of the interval. A
point is picked in the largest of the two intervals
$\left[\lambda_0,\lambda_1\right]$ or
$\left[\lambda_1,\lambda_2\right]$ and is used to reduce the
initial interval.

If $\lambda_1-\lambda_0\leq\lambda_2-\lambda_1$ we compute
$\lambda_4 =\lambda_1 + \omega\left(\lambda_2-\lambda_1\right)$
where $\omega$ is the golden
section\footnote{$\omega={3-\sqrt{5}\over 2}\approx0.38197$} from
Pythagorean lore. Choosing $\omega$ instead of $1/2$ ensures that
successive intervals have the same relative scale. A complete
derivation of this argument can be found in \cite{Press}. If
$\lambda_4$ yields a better function value than $\lambda_1$, the
new triplet of point becomes $\lambda_1$, $\lambda_4$ and
$\lambda_2$; otherwise, the triplet becomes $\lambda_0$,
$\lambda_1$ and $\lambda_4$.

If we have $\lambda_1-\lambda_0>\lambda_2-\lambda_1$ we compute
$\lambda_4 =\lambda_1 + \omega\left(\lambda_0-\lambda_1\right)$.
Then the new triplets can be either $\lambda_0$, $\lambda_4$ and
$\lambda_1$, or $\lambda_4$, $\lambda_1$ and $\lambda_2$.


The reader can verify that the interval decreases steadily
although not as fast as in the case of bisection where the
interval is halved at each iteration. Since the algorithm is using
the golden section it is called golden section search.

By construction the golden section search algorithm makes sure
that the optimum is always located between the points $\lambda_0$
and $\lambda_2$. Thus, at each iteration, the quantity
$\lambda_2-\lambda_0$ give an estimate of the error on the
position of the optimum.

\subsection{Optimizing in one dimension --- Smalltalk implementation}
\marginpar{Figure \ref{fig:soptimizingclasses} with the box {\bf
OneVariableFunctionOptimizer} grayed.} Listing
\ref{ls:optimizerOneDim} shows the class {\tt
DhbOneVariableFunctionOptimizer} implementing the search for an
optimum of a one-variable function using the golden section
search. The following code example shows how to use this class to
find the maximum of the gamma distribution discussed in section
\ref{sec:gammadist}.
\begin{codeExample}
\begin{verbatim}

 | distr finder maximum |
 distr := DhbGammaDistribution shape: 2 scale: 5.
 finder := DhbOneVariableFunctionOptimizer maximizingFunction: distr.
 maximum := finder evaluate.
\end{verbatim}
\end{codeExample}
The first line after the declarations creates a new instance of a
gamma distribution with parameters $\alpha = 2$ and $\beta = 5$.
The next line creates an instance of the optimum finder. The
selector used to create the instance selects a search for a
maximum. The last line is the familiar statement to evaluate the
iterations --- that is, performing the search for the maximum ---
and to retrieve the result. Since no initial value was supplied
the search begins at a random location.

The class {\tt DhbOneVariableFunctionOptimizer} is a subclass of
the class {\tt FunctionOptimizer}. It does not need any additional
instance variables. The golden section is kept as a class variable
and is calculated at the first time it is needed.

At each iteration the method {\tt nextXValue} is used to compute
the next position at which the function is evaluated. This
corresponding new optimizing point is added to the collection of
best points. Then, the method {\tt indexOfOuterPoint} is used to
determine which point must be discarded: it is always the second
point on either side of the best point. The precision of the
result is estimated from the bracketing interval in the method
{\tt computePrecision}, using relative precision (of course!).

The method {\tt addPoint:} of the superclass can be used to supply
an initial bracketing interval. The method {\tt
computeInitialValues} first checks whether a valid bracketing
interval has been supplied into the collection of best points. If
this is not the case, a search for a bracketing interval is
conducted using the class {\tt DhbOptimizingBracketFinder}
described in section \ref{sec:sbracket}. The instance of the
bracket finder is created with the method {\tt forOptimizer:} so
that its strategy and goal function are taken over from the golden
section optimum finder.

\begin{listing} Smalltalk golden section optimum finder \label{ls:optimizerOneDim}
$$\halign{ #\hfil&\quad#\hfil\cr {\sl Class}& {\Large\bf DhbOneVariableFunctionOptimizer}\cr
{\sl Subclass of }&{\tt DhbFunctionOptimizer}\cr\noalign{\vskip 1ex}

{\sl Class variable names:}&\parbox[t]{4 in}{\tt  GoldenSection }\cr\noalign{\vskip 1ex}}$$


Class methods
{\parskip 1ex\par\noindent}
{\bf defaultPrecision}
\begin{verbatim}
    ^DhbFloatingPointMachine new defaultNumericalPrecision * 10

\end{verbatim}
{\bf goldenSection}
\begin{verbatim}
    GoldenSection isNil ifTrue: [GoldenSection := (3 - 5 sqrt) / 2].
    ^GoldenSection

\end{verbatim}



Instance methods
{\parskip 1ex\par\noindent}
{\bf computeInitialValues}
\begin{verbatim}
    [ bestPoints size  > 3] whileTrue: [ bestPoints removeLast].
    bestPoints size = 3
        ifTrue: [ self hasBracketingPoints
                    ifFalse:[ bestPoints removeLast].
                ].
    bestPoints size < 3
        ifTrue: [ ( DhbOptimizingBracketFinder forOptimizer: self) 
                                                            evaluate].

\end{verbatim}
{\bf computePrecision}
\begin{verbatim}
    ^self precisionOf: ( ( bestPoints at: 2) position - ( bestPoints 
                                                  at: 3) position) abs
           relativeTo: ( bestPoints at: 1) position abs

\end{verbatim}
{\bf evaluateIteration}
\begin{verbatim}
    self addPointAt: self nextXValue.
    bestPoints removeAtIndex: self indexOfOuterPoint.
    ^self computePrecision

\end{verbatim}
{\bf hasBracketingPoints}
\begin{verbatim}
    | x1 |
    x1 := ( bestPoints at: 1) position.
    ^( ( bestPoints at: 2) position - x1) * (( bestPoints at: 3) 
                                                    position - x1) < 0

\end{verbatim}
{\bf indexOfOuterPoint}
\begin{verbatim}
    | inferior superior x |
    inferior := false.
    superior := false.
    x := bestPoints first position.
    2 to: 4 do:
        [ :n |
          ( bestPoints at: n) position < x
                ifTrue: [ inferior
                            ifTrue: [ ^n].
                          inferior := true.
                        ]
                ifFalse:[ superior
                            ifTrue: [ ^n].
                          superior := true.
                        ].
        ].

\end{verbatim}
{\bf nextXValue}
\begin{verbatim}
    | d3 d2 x1 |
    x1 := ( bestPoints at: 1) position.
    d2 := ( bestPoints at: 2) position - x1.
    d3 := ( bestPoints at: 3) position - x1.
    ^( d3 abs > d2 abs ifTrue: [ d3]
                       ifFalse:[ d2]) * self class goldenSection + x1

\end{verbatim}
{\bf reset}
\begin{verbatim}
    [ bestPoints isEmpty] whileFalse: [ bestPoints removeLast].

\end{verbatim}


\end{listing}

\subsection{Optimizing in one dimension --- Java implementation}
\marginpar{Figure \ref{fig:joptimizingclasses} with the box {\bf
OneVariableFunctionOptimizer} grayed.} Listing
\ref{lj:optimizerOneDim} shows the class {\tt
OneVariableFunctionOptimizer} implementing the search for an
optimum of a one-variable function using the golden section search
in Java. The following code example shows how to use this class to
find the maximum of the gamma distribution discussed in section
\ref{sec:gammadist}.
\begin{codeExample}
\begin{verbatim}

    CauchyDistribution distr = new CauchyDistribution( 10, 5);
    MaximizingPointFactory strategy = new MaximizingPointFactory();
    OneVariableFunctionOptimizer finder =
                new OneVariableFunctionOptimizer( distr, strategy);
    finder.setDesiredPrecision( 1.0e-5);
    finder.evaluate();
    double result = finder.getResult();
\end{verbatim}
\end{codeExample}
The first line creates an instance of a gamma distribution with
parameters $\alpha = 2$ and $\beta = 5$. The next line creates an
instance of a maximizing point factory. This factory is used as
the strategy for the optimum finder created on the next line, the
first argument of the constructor method being the function to
maximize. A medium precision is given explicitly since the default
provided by the class {\tt IterativeProcess} is likely to be
meaningless (\cf \cite{Press}). Since no initial value is
supplied, the search begins at a random point. After a call to the
method {\tt evaluate} --- common to all iterative processes
--- the result is retrieved on the last line.

\noindent The class {\tt OneVariableFunctionOptimizer} is a
subclass of the class {\tt FunctionalIterator} described in
section \ref{sec:jiterrel}. It has the following instance
variables
\begin{description}
  \item[\tt bestPoints] an array of 3 optimizing points corresponding
  to the values $\lambda_0$, $\lambda_1$ and $\lambda_2$ in this order;
  thus, the best is always the middle one;
  \item[\tt pointFactory] the factory used to create optimizing
  points (\cf section \ref{sec:jgeneralOpt}).
\end{description}
The private static variable {\tt goldenSection} holds the value of
the golden section.

The method {\tt evaluateIteration} determines which side of the
braketing interval the bisecting step will be performed. The
bisection step of the algorithm is performed within the method
{\tt reducePoints}. The arguments of the method is either 0 or 2
depending on the selected side. The precision returned by the
method {\tt evaluateIteration} is of course calculated using
relative precision.

The method {\tt initializeIterations} uses the optimum bracket
finder described in section \ref{sec:jbracket} to obtain the first
bracketing interval. The search for the interval is conducted from
the supplied initial value. The instance of the bracket finder is
created by supplying the same goal function and same strategy as
the initial golden section finder.

\begin{listing} Java implementation of the golden section optimum search
\label{lj:optimizerOneDim}
\input{Java/DhbOptimizing/OneVariableFunctionOptimizer.java}
\end{listing}

\section{Bracketing the optimum in one dimension}
\label{sec:bracket} As we have seen in section \ref{sec:optonedim}
the golden section algorithm requires the knowledge of a
bracketing interval. This section describes a very simple
algorithm to obtain a bracketing interval with certainty if the
function is continuous and does indeed have an optimum of the
sought type.

The algorithm goes as follows. Take two points $\lambda_0$ and
$\lambda_1$. If they do not exist, pick up some random values
(random generators are described in section \ref{sec:random}). Let
us assume that $g\left(\lambda_1\right)$ is better than
$g\left(\lambda_0\right)$.
\begin{enumerate}
  \item Let $\lambda_2=3\lambda_1-2\lambda_0$, that is, $\lambda_2$ is twice as far from $\lambda_1$ than
$\lambda_0$ and is located on the other side, toward the optimum.
  \item If $g\left(\lambda_1\right)$ is better than
$g\left(\lambda_2\right)$ we have a bracketing interval; the
algorithm is stopped.
  \item Otherwise, set $\lambda_0=\lambda_1$ and $\lambda_1=\lambda_2$ and go back to step 1.
\end{enumerate}
The reader can see that the interval
$\left[\lambda_0,\lambda_1\right]$ is increasing at each step.
Thus, if the function has no optimum of the sought type, the
algorithm will cause a floating overflow exception quite rapidly.

\noindent The implementation in each language have too little in
common. The common section is therefore omitted.

\subsection{Bracketing the optimum --- Smalltalk implementation}
\label{sec:sbracket} \marginpar{Figure
\ref{fig:soptimizingclasses} with the box {\bf
OptimizingBracketFinder} grayed.} Listing
\ref{ls:optimizerbracket} shows the Smalltalk code of the class
implementing the search algorithm for an optimizing bracket. The
class {\tt DhbOptimizingBracketFinder} is a subclass of class {\tt
DhbOneVariableFunctionOptimizer} from section
\ref{ls:optimizerOneDim}. This was a convenient, but not
necessary, choice to be able to reuse most of the management and
accessor methods. The methods pertaining to the algorithm are of
course quite different.

Example of use of the optimizing bracket finder can be found in
method {\tt computeInitialValues} of class {\tt
DhbOneVariableFunctionOptimizer} (\cf listing
\ref{ls:optimizerOneDim}).

The method {\tt setInitialPoints:} allows to use the collection of
best points of another optimizer inside the class. This breach to
the rule of hiding the implementation can be tolerated here
because the class {\tt DhbOptimizingBracketFinder} is used
exclusively with the class {\tt DhbOneVariableFunctionOptimizer}.
It allows the two class to use the same sorted collection of
optimizing points. If no initial point has been supplied, it is
obtained from a random generator.

The {\sl precision} calculated in the method {\tt
evaluateIteration} is a large number, which becomes negative as
soon as the condition to terminate the algorithm is met. Having a
negative precision causes an iterative process as defined in
chapter \ref{ch:iteration} to stop.

\begin{listing} Smalltalk optimum bracket finder \label{ls:optimizerbracket}
$$\halign{ #\hfil&\quad#\hfil\cr {\sl Class}& {\Large\bf DhbOptimizingBracketFinder}\cr
{\sl Subclass of }&{\tt DhbOneVariableFunctionOptimizer}\cr\noalign{\vskip 1ex}
}$$


Class methods
{\parskip 1ex\par\noindent}
{\bf initialPoints:} {\tt aSortedCollection} {\bf function:} {\tt aFunction}
\begin{verbatim}
    ^super new setInitialPoints: aSortedCollection; setFunction: 
                                                             aFunction

\end{verbatim}



Instance methods
{\parskip 1ex\par\noindent}
{\bf computeInitialValues}
\begin{verbatim}
    [bestPoints size < 2] whileTrue: [self addPointAt: Number random]

\end{verbatim}
{\bf evaluateIteration}
\begin{verbatim}
    | x1 x2 |
    x1 := ( bestPoints at: 1) position.
    x2 := ( bestPoints at: 2) position.
    self addPointAt: ( x1 * 3 - ( x2 * 2)).
    precision := ( x2 - x1) * ( ( bestPoints at: 3) position - x1).
    self hasConverged
        ifFalse:[ bestPoints removeLast].
    ^precision

\end{verbatim}
{\bf finalizeIterations}
\begin{verbatim}
    result := bestPoints.

\end{verbatim}
{\bf initializeForOptimizer:} {\tt aFunctionOptimizer}
\begin{verbatim}
    super initializeForOptimizer: aFunctionOptimizer.
    bestPoints := aFunctionOptimizer bestPoints.
    ^self

\end{verbatim}
{\bf setInitialPoints:} {\tt aSortedCollection}
\begin{verbatim}
    bestPoints := aSortedCollection.

\end{verbatim}


\end{listing}

\subsection{Bracketing the optimum --- Java implementation}
\label{sec:jbracket} \marginpar{Figure
\ref{fig:joptimizingclasses} with the box {\bf
OptimizingBracketFinder} grayed.} Listing
\ref{lj:optimizerbracket} shows the Smalltalk code of the class
implementing the search algorithm for an optimizing bracket. The
class {\tt OptimizingBracketFinder} is a subclass of the class
{\tt FunctionalIterator} defined in section \ref{sec:jiterrel}. It
has the same instance variables as the class {\tt
OneVariableFunctionOptimizer}.

The method {\tt initializeIterations} gets an initial value using
a random generator if no initial value has been supplied. The
other points of the initial interval are obtained from a random
generator. At the end of the method, the array of best points
contains 3 optimizing points sorted in ascending order of their
position.

The method {\tt evaluateIteration} first tests whether the optimum
seems to be located toward negative or positive values. After the
test the interval is expanded toward to corresponding direction by
one of the methods {\tt moveTowardNegative} or  {\tt
moveTowardPositive}.

The method {\tt evaluateIteration} calcuate a pseudo precision: it
is a large number, which becomes negative as soon as the condition
to terminate the algorithm is met. Having a negative precision
causes an iterative process as defined in chapter
\ref{ch:iteration} to stop.

The object using the bracket finder can retrieve the array of best
points using the method {\tt getBestPoints}. This method
constitutes a breach to the rule of hiding the implementation. In
this case, however, this is an acceptable breach: the class {\tt
OneVariableFunctionOptimizer} can retrieve the initial bracketing
interval and use it immediately without having to evaluate again
the function at these positions. After all, the bracketing
interval {\sl is} the result of the optimizing bracket search.

\begin{listing} Java optimum bracket finder
\label{lj:optimizerbracket}
\input{Java/DhbOptimizing/OptimizingBracketFinder.java}
\end{listing}

\section{Powell's algorithm}
\label{sec:powell}Powell's algorithm is one of many hill climbing
algorithms \cite{Press}. The idea underlying Powell's algorithm is
that once an optimum has been found in one direction, the chance
for the biggest improvement lies in the direction perpendicular to
that direction. A mathematical formulation of this sentence can be
found in \cite{Press} and references therein. Powell's algorithm
provides a way to keep track of the next best direction at each
iteration step.

\noindent The original steps of Powell's algorithm are as follow:
\begin{enumerate}
  \item Let ${\bf x}_0$ the best point so far and initialize a
  series of vectors ${\bf v}_1,\ldots,{\bf v}_n$ forming the system
  of reference ($n$ is the
  dimension of the vector ${\bf x}_0$); in
  other words the components of the vector ${\bf v}_k$ are all
  zero except for the $k\th$ components, which is one.
  \item Set $k=1$.
  \item Find the optimum of the goal function along the direction ${\bf
  v}_k$ starting from point ${\bf x}_{k-1}$. Let ${\bf x}_k$  be
  the position of that optimum.
  \item Set $k=k+1$. If $k\leq n$, go back to step 3.
  \item For $k=1,\ldots,n-1$, set ${\bf v}_k={\bf v}_{k-1}$.
  \item Set ${\bf v}_n ={ {\bf x}_n-{\bf x}_0 \over \left| {\bf x}_n-{\bf x}_0\right|}$.
  \item Find the optimum of the goal function along the direction ${\bf
  v}_n$. Let ${\bf x}_{n+1}$  be the position of that optimum.
  \item If $\left| {\bf x}_n-{\bf x}_0\right|$ is less than the
  desired precision, terminate.
  \item Otherwise, set ${\bf x}_0={\bf x}_{n+1}$ and go back to
  step 1.
\end{enumerate}
There is actually two hill climbing algorithms within each other.
The progression obtained by the inner loop is taken as the
direction in which to continue the search.

Powell recommends to use this algorithm on goal functions having a
quadratic behaviour near the optimum. It is clear that this
algorithm cannot be used safely on any function. If the goal
function has narrow valleys, all directions ${\bf v}_1,\ldots,{\bf
v}_n$ will become colinear when the algorithm ends up in such a
valley. Thus, the search is likely to end up in a position where
no optimum is located. Press et al. \cite{Press} mention two
methods avoiding such problems: one method is quite complex and
the other slows down the convergence of the algorithm.

In spite of this {\it caveat}, we implement Powell's algorithm in
its original form. However, we recommend its use only in the
vicinity of the minimum. In section \ref{sec:multistrategy} we
show how other techniques can be utilized to read the vicinity of
the optimum, where Powell's algorithm can safely be used to make
the final determination of the optimum's position.

\subsection{Powell's algorithm --- General implementation}
Since the class implementing the vector projected function
$g\left(\lambda\right)$ described in sections
\ref{sec:sgeneralOpt} and \ref{sec:jgeneralOpt} keep the vector
${\bf x}_0$ and ${\bf v}$ in instance variables, there is no need
to allocate explicit storage for the vectors ${\bf
x}_1,\ldots,{\bf x}_n$ and ${\bf v}_1,\ldots,{\bf v}_n$. Instead,
the class implementing Powell's algorithm keep an array of vector
projected functions with the corresponding parameters. Then, the
manipulation of the vector ${\bf x}_1,\ldots,{\bf x}_n$ and ${\bf
v}_1,\ldots,{\bf v}_n$ is made directly on the projected function.

Since the origin of the projected function is always the starting
value, ${\bf x}_k$, the initial value for the search of the
optimum of the function $g\left(\lambda\right)$ is always 0.

The method {\tt initializeIterations} allocated a series of vector
projected functions starting with the axes of the reference
system. This method also creates an instance of a one dimensional
optimum finder kept in the instance variable, {\tt
unidimensionalFinder}. The goal function of the finder is
alternatively assigned to each of the projected functions.

We made a slight modification to Powell's algorithm. If the norm
of the vector ${\bf x}_n-{\bf x}_0$ at step 6 is smaller than the
desired precision, the directions are only rotated, the assignment
of step 6 is not done and the search of step 7 is omitted.

The precision computed  at the end of each iterations is the
maximum of the relative change on all components between the
vectors ${\bf x}_n$ and ${\bf x}_0$.

\subsection{Powell's algorithm --- Smalltalk implementation}
\marginpar{Figure \ref{fig:soptimizingclasses} with the box {\bf
HillClimbingOptimizer} grayed.} Listing \ref{ls:optimizerpowell}
shows the implementation of Powell's algorithm in Smalltalk. The
following code example shows how to find the maximum of a vector
function
\begin{codeExample}
\label{ex:spowell}
\begin{verbatim}

 | fBlock educatedGuess hillClimber result |
\end{verbatim}
 {\tt fBlock :=<\sl the goal function\tt >}\hfil\break
 {\tt educatedGuess :=<\sl a vector not too far from the optimum\tt >}
\begin{verbatim}
 hillClimber := DhbHillClimbingOptimizer maximizingFunction: fBlock.
 hillClimber initialValue: educatedGuess.
 result := hillClimber evaluate.
\end{verbatim}
\end{codeExample}
The class {\tt DhbHillClimbingOptimizer} is a subclass of class
{\tt DhbFunctionOptimizer}. It has only one additional instance
variable, {\tt unidimensionalFinder}, to hold the one-dimensional
optimizer used to find an optimum of the goal function along a
given direction.

The method {\tt evaluateIteration} uses the method {\tt
inject:into:} to perform steps 2-4 of the algorithm. Similarly
step 5 of the algorithm is performed with a method {\tt
inject:into:} within the method {\tt shiftDirection}. This mode of
using the iterator method {\tt inject:into:} --- performing an
action involving two consecutive elements of an indexed collection
--- is somewhat unusual, but highly convenient\cite{Beck}. The method {\tt
minimizeDirection:} implements step 3 of the algorithm.

\begin{listing} Smalltalk implementation of Powell's algorithm
\label{ls:optimizerpowell}
$$\halign{ #\hfil&\quad#\hfil\cr {\sl Class}& {\Large\bf DhbHillClimbingOptimizer}\cr
{\sl Subclass of }&{\tt DhbFunctionOptimizer}\cr\noalign{\vskip 1ex}

{\sl Instance variable names:}&\parbox[t]{4 in}{\tt  unidimensionalFinder }\cr\noalign{\vskip 1ex}}$$


Instance methods
{\parskip 1ex\par\noindent}
{\bf computeInitialValues}
\begin{verbatim}
    unidimensionalFinder := DhbOneVariableFunctionOptimizer
                                                   forOptimizer: self.
    unidimensionalFinder desiredPrecision: desiredPrecision.
    bestPoints := (1 to: result size)
                 collect: [ :n | ( DhbVectorProjectedFunction
                                      function: functionBlock)
                                     direction: ((DhbVector
                                                     new: result size)
                                                         atAllPut: 0;
                                                         at: n put: 1;
                                                         yourself);
                                            yourself
                          ].

\end{verbatim}
{\bf evaluateIteration}
\begin{verbatim}
    | oldResult |
    precision := 1.
    bestPoints inject: result
                 into: [ :prev :each | ( self minimizeDirection: each
                                                         from: prev)].
    self shiftDirections.
    self minimizeDirection: bestPoints last.
    oldResult := result.
    result := bestPoints last origin.
    precision := 0.
    result with: oldResult do:
        [ :x0 :x1 |
          precision := ( self precisionOf: (x0 - x1) abs relativeTo:
                                               x0 abs) max: precision.
        ].
    ^ precision
\end{verbatim}
{\bf minimizeDirection:} {\tt aVectorFunction}
\begin{verbatim}
    ^ unidimensionalFinder
        reset;
        setFunction: aVectorFunction;
        addPointAt: 0;
        addPointAt: precision;
        addPointAt: precision negated;
        evaluate
\end{verbatim}
{\bf minimizeDirection:} {\tt aVectorFunction} {\bf from:} {\tt aVector}
\begin{verbatim}
Function from: aVector
    ^aVectorFunction
        origin: aVector;
        argumentWith: (self minimizeDirection: aVectorFunction)
\end{verbatim}
{\bf shiftDirections}
\begin{verbatim}
    | position delta firstDirection |
    firstDirection := bestPoints first direction.
    bestPoints inject: nil
                    into: [ :prev :each |
                            position isNil
                                ifTrue: [ position := each origin]
                                ifFalse:[ prev direction: each
                                                           direction].
                            each
                            ].
    position := bestPoints last origin - position.
    delta := position norm.
    delta > desiredPrecision
        ifTrue: [ bestPoints last direction: (position scaleBy: (1 /
                                                              delta))]
        ifFalse:[ bestPoints last direction: firstDirection].
\end{verbatim}

\end{listing}

\subsection{Powell's algorithm --- Java implementation}
\marginpar{Figure \ref{fig:joptimizingclasses} with the box {\bf
HillClimbingOptimizer} grayed.} Listing \ref{lj:optimizerpowell}
shows the implementation of Powell's algorithm in Java. The
following code example shows how to use this class to find the
maximum of a vector function.
\begin{codeExample}
\label{ex:jpowell}
\begin{verbatim}
\end{verbatim}
{\tt ManyVariableFunction func = <\sl the goal function\tt
>}\hfil\break
 {\tt double[] educatedGuess =<\sl an array of double not too far from the optimum\tt >}
\begin{verbatim}
    MaximizingPointFactory strategy = new MaximizingPointFactory();
    HillClimbingOptimizer hillClimber =
                new HillClimbingOptimizer( func, strategy);
    hillClimber.setInitialValue( educatedGuess);
    hillClimber.evaluate();
    double[] result = hillClimber.getResult();
\end{verbatim}
\end{codeExample}

The class {\tt HillClimbingOptimizer} is a subclass of class {\tt
MultiVariableOptimizer}. It has the following instance variables
\begin{description}
  \item[\tt unidimensionalFinder] an instance of the class {\tt
  OneVariableFunctionOptimizer}; this instance is used to find the
  optimum of the function in steps 3 and 7 of the algorithm;
  \item[\tt projections] an array of projected functions.
\end{description}

\begin{listing} Java implementation of Powell's algorithm
\label{lj:optimizerpowell}
\input{Java/DhbOptimizing/HillClimbingOptimizer.java}
\end{listing}

\section{Simplex algorithm}
\label{sec:simplex} The simplex algorithm, invented by Nelder and
Mead, provides an efficient way to find a good approximation of
the optimum of a function starting from any place \cite{Press}.
The only trap into which the simplex algortihm can run into is a
local optimum. On the other hand, this algorithm does not converge
very well in the vicinity of the optimum. Thus, it must not be
used with the desired precision set to a very low value. Once the
optimum has been found with the simplex algorithm, other more
precise algorithms can then be used, such as the ones describes in
section \ref{sec:optimum} or \ref{sec:powell}. MINUIT uses a
combination of simplex and Newton algorithms. Our implementation
of general purpose optimizer uses a combination of simplex and
Powell algorithms.

A simplex in a $n$-dimensional space is a figure formed with $n+1$
summits. For example, a simplex in a 2-dimensional space is a
triangle; a simplex in a 3- dimensional space is a tetrahedron.
Let us now discuss the algorithm for finding the optimum of a
function with a simplex.
\begin{enumerate}
  \item Pick up $n+1$ points in the search space and evaluate
  the goal function at each of them. Let ${\bf A}$ be the summit
  yielding the worst function's value.
  \item if the size of the simplex is smaller than the desired
  precision, terminate the algorithm.
  \item Calculate ${\bf G}$, the center of gravity of the $n$ best points,
  that is all points except ${\bf A}$.
  \item Calculate the location of the symmetric point of ${\bf A}$
  relative to ${\bf G}$: ${\bf A}^{\prime}=2{\bf G}-{\bf A}$.
  \item If $f\left({\bf A}^{\prime}\right)$ is not the best value
  found so far go to step 9.
  \item Calculate the point ${\bf A}^{\prime\prime}=2{\bf A}^{\prime}-{\bf
  G}$, that is a point twice as far from ${\bf G}$ as ${\bf
  A}^{\prime}$.
  \item If $f\left({\bf A}^{\prime\prime}\right)$ is a
  better value than $f\left({\bf A}^{\prime}\right)$ build a new
  simplex with the point ${\bf A}$ replaced by the point ${\bf
  A}^{\prime\prime}$ and go to step 2.
  \item Otherwise, build a new
  simplex with the point ${\bf A}$ replaced by the point ${\bf
  A}^{\prime}$ and go to step 2.
  \item Calculate the point ${\bf B}= {\displaystyle\left({\bf G}+{\bf
A}\right)\over\displaystyle 2}$.
  \item If $f\left({\bf B}\right)$ yields the best value found so far
  build a new simplex with the point ${\bf A}$ replaced by the point ${\bf
  A}^{\prime\prime}$ and go to step 2.
  \item Otherwise build a new simplex obtained by dividing all
  edges leading to the point yielding the best value by 2 and go
  back to step 2.
\end{enumerate}
Figure \ref{fig:simplexsample} shows the meaning of the operations
involved in the algorithm in the 3 dimensional case.
\begin{figure}
\center\psfig{figure=Figures/Simplex.eps, width=10cm}
\caption{Operations of the simplex
algorithm}\label{fig:simplexsample}
\end{figure}
Step 6 makes the simplex grow into the direction where the
function is the best so far. Thus, the simplex becomes elongated
in the expected direction of the optimum. Because of its
geometrical shape, the next step is necessarily taken along
another direction, causing an exploration of the regions
surrounding the growth obtained at the preceding step. Over the
iterations, the shape of the simplex adapts itself to narrow
valleys where the hill climbing algorithms notoriously get into
trouble. Steps 9 and 11 ensures the convergence of the algorithm
when the optimum lies inside the simplex. In this mode the simplex
works very much like the golden section search or the bisection
algorithms.

Finding the initial points can be done in several ways. If a good
approximation of the region where the maximum might be located can
be obtained one uses that approximation as a start and generate
$n$ other points by finding the optimum of the function along each
axis. Otherwise, one can generate random points and select $n+1$
points yielding the best values to build the initial simplex. In
all cases, one must make sure that the initial simplex has a
non-vanishing size in all dimensions of the space. Otherwise the
algorithm will not reach the optimum.

\subsection{Simplex algorithm --- General implementation}
The class implementing the simplex algorithm belong to the
hierarchy of the iterative processes discussed in chapter
\ref{ch:iteration}. The method {\tt evaluateIteration} directly
implements the steps of the algorithm as described above. The
points ${\bf G}$, ${\bf A}^{\prime}$, ${\bf A}^{\prime\prime}$ and
${\bf B}$ are calculated using the vector operations described in
section \ref{sec:linearalgebra}.

The routine {\tt initializeIterations} assumes that an initial
value has been provided. It then finds the location of an optimum
of the goal function along each axis of the reference system
starting each time from the supplied initial value, unlike hill
climbing algorithms. Restarting from the initial value is
necessary to avoid creating a simplex with a zero volume. Such
mishaps can arise when the initial value is located on an axis of
symmetry of the goal function. This can happen quite frequently
with {\sl educated guesses}.

\subsection{Simplex algorithm --- Smalltalk implementation}
\marginpar{Figure \ref{fig:soptimizingclasses} with the box {\bf
SimplexOptimizer} grayed.} Listing \ref{ls:optimizersimplex} shows
the Smalltalk implementation of the simplex algorithm. The
following code example shows how to invoke the class to find the
minimum of a vector function.
\begin{codeExample}
\begin{verbatim}

 | fBlock educatedGuess simplex result |
\end{verbatim}
 {\tt fBlock :=<\sl the goal function\tt >}\hfil\break
 {\tt educatedGuess :=<\sl a vector in the search space\tt >}
\begin{verbatim}
 simplex := DhbSimplexOptimizer minimizingFunction: fBlock.
 simplex initialValue: educatedGuess.
 result := simplex evaluate.
\end{verbatim}
\end{codeExample}
Except for the line creating the instance of the simplex
optimizer, this code example is identical to the example of
Powell's hill climbing algorithm (code example \ref{ex:spowell}).

The class {\tt DhbSimplexOptimizer} is a subclass of class {\tt
DhbFunctionOptimizer}. In order to be able to use the iterator
methods efficiently, the worst point of the simplex, ${\bf A}$, is
held in a separate instance variable {\tt worstPoint}. As we do
not need to know the function's value $f\left({\bf A}\right)$, it
is kept as a vector. The remaining points of the simplex are kept
in the instance variable {\tt bestPoints} of the superclass. Since
this collection is sorted automatically when points are inserted
to it, there is no explicit sorting step.

\begin{listing} Smalltalk implementation of simplex algorithm
\label{ls:optimizersimplex}
$$\halign{ #\hfil&\quad#\hfil\cr {\sl Class}& {\Large\bf DhbSimplexOptimizer}\cr
{\sl Subclass of }&{\tt DhbFunctionOptimizer}\cr\noalign{\vskip 1ex}

{\sl Instance variable names:}&\parbox[t]{4 in}{\tt  worstVector }\cr\noalign{\vskip 1ex}}$$


Class methods
{\parskip 1ex\par\noindent}
{\bf defaultPrecision}
\begin{verbatim}
    ^DhbFloatingPointMachine new defaultNumericalPrecision * 1000

\end{verbatim}



Instance methods
{\parskip 1ex\par\noindent}
{\bf buildInitialSimplex}
\begin{verbatim}
    | projectedFunction finder partialResult |
    projectedFunction := DhbProjectedOneVariableFunction 
                function: functionBlock.
    finder := DhbOneVariableFunctionOptimizer forOptimizer: self.
    finder setFunction: projectedFunction.
    [bestPoints size < (result size + 1)] whileTrue: 
            [projectedFunction
                setArgument: result;
                bumpIndex.
            partialResult := finder
                        reset;
                        evaluate.
            bestPoints add: (optimizingPointClass 
                        vector: (projectedFunction argumentWith: 
                                                        partialResult)
                        function: functionBlock)]

\end{verbatim}
{\bf computeInitialValues}
\begin{verbatim}
    bestPoints 
        add: (optimizingPointClass vector: result function: 
                                                       functionBlock).
    self buildInitialSimplex.
    worstVector := bestPoints removeLast position

\end{verbatim}
{\bf computePrecision}
\begin{verbatim}
    | functionValues bestFunctionValue |
    functionValues := bestPoints collect: [ :each | each value].
    bestFunctionValue := functionValues removeFirst.
    ^functionValues inject: 0
                    into: [ :max :each | ( self precisionOf: ( each - 
   bestFunctionValue) abs relativeTo: bestFunctionValue abs) max: max]

\end{verbatim}
{\bf contract}
\begin{verbatim}
    | bestVector oldVectors |
    bestVector := bestPoints first position.
    oldVectors := OrderedCollection with: worstVector.
    [bestPoints size > 1] whileTrue: [oldVectors add: bestPoints 
                                                 removeLast position].
    oldVectors do: [:each | self contract: each around: bestVector].
    worstVector := bestPoints removeLast position.
    ^self computePrecision

\end{verbatim}
{\bf contract:} {\tt aVector} {\bf around:} {\tt bestVector}
\begin{verbatim}
    bestPoints 
        add: (optimizingPointClass vector: bestVector * 0.5 + 
                                                       (aVector * 0.5)
                function: functionBlock)

\end{verbatim}
{\bf evaluateIteration}
\begin{verbatim}
    | centerOfGravity newPoint nextPoint |
    centerOfGravity := (bestPoints inject: ((worstVector copy)
                        atAllPut: 0;
                        yourself)
                into: [:sum :each | each position + sum]) * (1 / 
                                                     bestPoints size).
    newPoint := optimizingPointClass vector: 2 * centerOfGravity - 
                                                           worstVector
                function: functionBlock.
    (newPoint betterThan: bestPoints first) 
        ifTrue: 
            [nextPoint := optimizingPointClass 
                        vector: newPoint position * 2 - 
                                                       centerOfGravity
                        function: functionBlock.
            (nextPoint betterThan: newPoint) ifTrue: [newPoint := 
                                                           nextPoint]]
        ifFalse: 
            [newPoint := optimizingPointClass 
                        vector: centerOfGravity * 0.666667 + 
                                              (worstVector * 0.333333)
                        function: functionBlock.
            (newPoint betterThan: bestPoints first) ifFalse: [^self 
                                                           contract]].
    worstVector := bestPoints removeLast position.
    bestPoints add: newPoint.
    result := bestPoints first position.
    ^self computePrecision

\end{verbatim}
{\bf printOn:} {\tt aStream}
\begin{verbatim}
    super printOn: aStream.
    aStream cr. 
    worstVector printOn: aStream.

\end{verbatim}


\end{listing}
\subsection{Simplex algorithm --- Java implementation}
\marginpar{Figure \ref{fig:joptimizingclasses} with the box {\bf
SimplexOptimizer} grayed.} Listing \ref{ls:optimizersimplex} shows
the Java implementation of the simplex algorithm. The following
code example shows how to invoke the class to find the minimum of
a vector function.
\begin{codeExample}
\begin{verbatim}
\end{verbatim}
{\tt ManyVariableFunction func = <\sl the goal function\tt
>}\hfil\break
 {\tt double[] educatedGuess =<\sl an array of double representing one point in the search space\tt >}
\begin{verbatim}
    MinimizingPointFactory strategy = new MinimizingPointFactory();
    SimplexOptimizer simplex = new SimplexOptimizer( func, strategy);
    simplex.setInitialValue( educatedGuess);
    simplex.evaluate();
    double[] result = simplex.getResult();
\end{verbatim}
\end{codeExample}
Except for the lines creating the strategy and the optimizer, this
code example is identical to that of the Powell's algorithm (code
example \ref{ex:jpowell}).

The class {\tt SimplexOptimizer} is a subclass of class {\tt
MultiVariableOptimizer}. The additional instance variable {\tt
simplex} contains an array of optimizing vectors whose positions
are the summit of the simplex.

At the end of the contraction operation, the best points must be
sorted --- using the method {\tt sortPoints} of the superclass ---
because this transformation can alter the order of the points
considerably. In all other case, one has found the best point so
far; the method {\tt addBestPoint} is then used to discard to
worst point, shift the remaining points and add the best point in
the first position of the array {\tt simplex}.

\begin{listing} Java implementation of simplex algorithm
\label{lj:optimizersimplex}
\input{Java/DhbOptimizing/SimplexOptimizer.java}
\end{listing}

\section{Genetic algorithm}
All optimizing algorithm discussed so far have one common flaw:
they all terminate when a local optimum is encountered. In most
problems, however, one wants to find the absolute optimum of the
function. This is especially true if the goal function represents
some economical merit.

\noindent One academic example is the maximization of the function
\begin{equation}
\label{eq:geneticCase}
  f\left({\bf x}\right)={ \sin^2\left|{\bf x}\right| \over\left|{\bf
  x}\right|^2}.
\end{equation}
This function has an absolute maximum at ${\bf x}=0$, but all
algorithms discussed so far will end up inside a ring
corresponding to $\left|{\bf x}\right|=n\pi/2$ where $n$ is any
positive odd integer.

In 1975 John Holland introduced a new type of algorithm --- dubbed
genetic algorithm --- because it tries to mimic the evolutionary
process identified as the cause for the diversity of living
species by Charles Darwin. In a genetic algorithm the elements of
the search space are considered as the chromosomes of individuals;
the goal function is considered as the measure of the fitness of
the individual to adapt itself to its
environment\cite{BerLin}\cite{Koza}. The iterations are aping (pun
intended) the Darwinian principle of survival and reproduction. At
each iteration, the fittest individuals survive and reproduce
themselves. To bring some variability to the algorithm mutation
and crossover of chromosomes are taken into account.

Mutation occurs when one gene of a chromosome is altered at
reproduction time. Crossover occurs when two chromosomes break
themselves and recombine with the piece coming from the other
chromosome. These processes are illustrated on figure
\ref{fig:crossover}.
\begin{figure}
\center\psfig{figure=Figures/Crossover.eps, width=11cm}
\caption{Mutation and crossover reproduction of
chromosomes}\label{fig:crossover}
\end{figure}
The point where the chromosomes are breaking is called the
crossover point. Which individual survives and reproduces itself,
when and where mutation occurs and when and where a crossover
happens is determined randomly. This is precisely the random
nature of the algorithm which gives it the ability to jump out of
a local optimum to look further for the absolute optimum.

\rubrique{Mapping the search space on chromosomes} To be able to
implement a genetic algorithm one must establish how to represent
the genes of a chromosome. At the smallest level the genes could
be the bits of the structure representing the chromosome. If the
search space of the goal function do cover the domain generated by
all possible permutations of the bits, this is a good approach.
However, this is not always a practical solution since some bit
combinations may be forbidden by the structure. For example, some
of the combinations of a 64 bit word do not correspond to a valid
floating point number.

In the case of the optimization of a vector function, the simplest
choice is to take the components of the vector as the genes.
Genetic algorithms are used quite often to adjust the parameters
of a neural network \cite{BerLin}. In this case, the chromosomes
are the coefficients of each neuron. Chromosomes can even be
computer subprograms in the case of genetic programming
\cite{Koza}. In this latter case, each individual is a computer
program trying to solve a given problem.

Figure \ref{fig:geneticFlow} shows a flow diagram of a general
genetic algorithm.
\begin{figure}
\center\psfig{figure=Figures/GeneticFlow.eps, width=10cm}
\caption{General purpose genetic algorithm}\label{fig:geneticFlow}
\end{figure}
The reproduction of the individual is taken literally: a copy of
the reproducing individual is copied into the next generation. The
important feature of a generic algorithm is that the building of
the next generation is a random process. To ensure the survival of
the fittest, the selection of the parents of the individuals of
the next generation is performed at random with uneven
probability: the fittest individuals have a larger probability of
being selected than the others. Mutation enables the algorithm to
create individuals having genes corresponding to unexplored
regions of the search space. Most of the times such mutants will
be discarded at the next iteration; but, in some cases, a mutation
may uncover a better candidate. In the case of the function of
equation \ref{eq:geneticCase}, this would correspond to jumping
from one ring to another ring closer to the function's maximum.
Finally the crossover operation mixes good genes in the hope of
building a better individual out of the properties of good
inidividuals. Like mutation the crossover operation gives a
stochastic behavior to the algorithm enabling it to explore
uncharted regions of the search space.

\note{Because of its stochastic nature a genetic algorithm is the
algorithm of choice when the goal function is expressed on
integers.}

\subsection{Genetic algorithm --- General implementation}
\label{sec:gengenetic} The left hand side of the diagram of figure
\ref{fig:geneticFlow} is quite similar to the flow diagram of an
iterative process (\cf figure \ref{fig:itercoarse} in chapter
\ref{ch:iteration}). Thus, the class implementing the genetic
algorithm is a subclass of the iterative process class discussed
in chapter \ref{ch:iteration}.

The {\sl genetic} nature of the algorithm is located in the right
hand side of the diagram of figure \ref{fig:geneticFlow}. As we
have mentioned before the implementation of the chromosomes is
highly problem dependent. All operations located in the top
portion of the mentioned area can be expressed in generic terms
without any knowledge of the chromosomic implementation. to handle
the lower part of the right hand side of the diagram of figure
\ref{fig:geneticFlow}, we shall implement a new object, the
chromosome manager.

One should also notice that the value of the function is not
needed when the next generation is build. Thus, the chromosome
manager does not need to have any knowledge of the goal function.
The goal function comes into play when transfering the next
generation to the {\sl mature} population, that is, the population
used for reproduction at the next iteration . At the maturity
stage, the value of the goal function is needed to identify the
fittest individuals. In our implementation, the next generation is
maintained by the chromosome manager whereas the population of
mature individuals is maintained by the object in charge of the
genetic algorithm which has the knowledge of the goal function.

\noindent The chromosome manager has the following instance
variables:
\begin{description}
  \item[\tt populationSize] contains the size of the population;
  one should pick up a large enough number to be able to cover the
  search space efficiently: the larger the dimension of the space
  search space, the larger must be the population size;
  \item[\tt rateOfMutation] contains the probability of having a
  mutation while reproducing;
  \item[\tt rateOfCrossover] contains the probability of having a
  crossover while reproducing.
\end{description}
All of these variables have getter and setter accessor methods. In
addition a convenience instance creation method is supplied to
create a chromosome manager with given values for all three
instance variables. The chromosome manager implements the
following methods:
\begin{description}
  \item[\tt isFullyPopulated] to signal that a sufficient number of individuals
  has been generated into the population;
  \item[\tt process] to process a pair of individuals; this method
  does the selection of the genetic operation and applies it;
  individuals are processed by pair to always have a possibility
  of crossover;
  \item[\tt randomnizePopulation] to generate a random population;
  \item[\tt reset] to create an empty population for the next
  generation.
\end{description}
Finally the chromosome manager must also implement methods
performing each of the genetic operations: reproduction, mutation
and crossover. The Smalltalk implementation supplies methods that
returns a new individual; the Java implementation supplies methods
that add a new individual to the population. The reason for this
difference come from the static typing requirements of Java.

The genetic optimizer is the object implementing the genetic
algorithm proper. It is a subclass of the iterative process class
described in chapter \ref{sec:iterrel}. In addition to the
handling of the iterations the genetic optimizer implements the
steps of the algorithm drawn on the top part of the right hand
side of the diagram of figure \ref{fig:geneticFlow}. It has one
instance variable containing the chromosome manager with which it
will interact. The instance creation method take three arguments:
the function to optimize, the optimizing strategy and the
chromosome manager.

The method {\tt initializeIteration} asks the chromosome manager
to supply a random population. The method {\tt evaluateIteration}
performs the loop of the right hand side of the diagram of figure
\ref{fig:geneticFlow}. It selects a pair of parents on which the
chromosome manager performs the genetic operation.

Selecting the genetic operation is performed with a random
generator. The values of the goal function are used as weights.
Let $f\left(p_i\right)$ be the value of the goal function for
individual $p_i$ and let $p_b$ and $p_w$ be respectively the
fittest and the lest fit individual found so far ($b$ stands for
best and $w$ stands for worst). One first computes the
unnormalized probability:
\begin{equation}
  \tilde{P}_i={\displaystyle f\left(p_i\right) - f\left(p_w\right)
  \over\displaystyle f\left(p_b\right) - f\left(p_w\right)}.
\end{equation}
This definition ensures that $\tilde{P}_i$ is always comprised
between 0 and 1 for any goal function. Then we can use the
discrete probability
\begin{equation}
\label{eq:geneticprob}
  P_i={\displaystyle  1
  \over\displaystyle \sum\tilde{P}_i} \tilde{P}_i.
\end{equation}
The sum in equation \ref{eq:geneticprob} is taken over the entire
population. An attentive reader will notice than this definition
assigns a zero probability of selecting the worst individuals.
This gives a slight bias to our implementation compared to the
original algorithm. This is can be easily compensated by taking a
sufficiently large population. The method {\tt randomScale}
calculates the $P_i$ of equation \ref{eq:geneticprob} and returns
an array containing the integrated sums:
\begin{equation}
  R_i=\sum_{k=0}^i P_i.
\end{equation}
The array $R_i$ is used to generate a random index to select
individuals for reproduction.

\noindent The transfer between the next generation and the mature
population is performed by the method {\tt collectPoints}.

In the general case, there is no possibility to decide when the
terminate the algorithm. In practice, it is possible that the
population stays stable for quite a while until suddenly a new
individual is found to be better than the rest. Therefore a
criteria based on the stability of the first best points is likely
to be beat the purpose of the algorithm, namely to jump out of a
local optimum. Some problems can define a threshold at which the
goal function is considered sufficiently good. In this case, the
algorithm can be stopped as soon as the value of the goal function
for the fittest individual becomes better than that threshold. In
the general case, however, the implementation of the genetic
algorithm simply returns a constant pseudo precision
--- set to one --- and runs until the maximum number of iterations
becomes exhausted.

\subsection{Genetic algorithm --- Smalltalk implementation}
\marginpar{Figure \ref{fig:soptimizingclasses} with the boxes {\bf
GeneticOptimizer}, {\bf ChromosomeManager} and {\bf
VectorChromosomeManager} grayed.} Listing \ref{ls:chromosome}
shows the code of an abstract chromosome manager in Smalltalk and
of a concrete implementation for vector chromosomes. The class
{\tt DhbChromosomeManager} has one instance variable in addition
to the variables listed in section \ref{sec:gengenetic}: {\tt
population}. This variable is an instance of an {\tt
OrderedCollection} containing the individuals of the next
generation being prepared.

The class {\tt DhbVectorChromosomeManager} is a sublcass of class
{\tt DhbChromosomeManager} implementing vector chromosomes. It has
two instance variables
\begin{description}
  \item[\tt origin] a vector containing the minimum possible
  values of the generated vectors;
  \item[\tt range] a vector containing the range of the generated
  vectors.
\end{description}
In other words {\tt origin} and {\tt range} are delimiting an
hypercube defining the search space.

\begin{listing} Smalltalk chromosome: abstract and concrete \label{ls:chromosome}
$$\halign{ #\hfil&\quad#\hfil\cr {\sl Class}& {\Large\bf DhbChromosomeManager}\cr
{\sl Subclass of }&{\tt Object}\cr\noalign{\vskip 1ex}

{\sl Instance variable names:}&\parbox[t]{4 in}{\tt  population populationSize rateOfMutation rateOfCrossover }\cr\noalign{\vskip 1ex}}$$


Class methods
{\parskip 1ex\par\noindent}
{\bf new:} {\tt anInteger} {\bf mutation:} {\tt aNumber1} {\bf crossover:} {\tt aNumber2}
\begin{verbatim}
    ^self new populationSize: anInteger; rateOfMutation: aNumber1; 
                                   rateOfCrossover: aNumber2; yourself

\end{verbatim}



Instance methods
{\parskip 1ex\par\noindent}
{\bf clone:} {\tt aChromosome}
\begin{verbatim}
    ^aChromosome copy

\end{verbatim}
{\bf crossover:} {\tt aChromosome1} {\bf and:} {\tt aChromosome2}
\begin{verbatim}
    ^self subclassResponsibility

\end{verbatim}
{\bf isFullyPopulated}
\begin{verbatim}
    ^population size >= populationSize 

\end{verbatim}
{\bf mutate:} {\tt aChromosome}
\begin{verbatim}
    ^self subclassResponsibility

\end{verbatim}
{\bf population}
\begin{verbatim}
    ^population

\end{verbatim}
{\bf populationSize:} {\tt anInteger}
\begin{verbatim}
    populationSize := anInteger.

\end{verbatim}
{\bf process:} {\tt aChromosome1} {\bf and:} {\tt aChromosome2}
\begin{verbatim}
    | roll |
    roll := Number random.
    roll < rateOfCrossover 
        ifTrue: [population addAll: (self crossover: aChromosome1 
                                                   and: aChromosome2)]
        ifFalse: 
            [roll < (rateOfCrossover + rateOfMutation) 
                ifTrue: 
                    [population
                        add: (self mutate: aChromosome1);
                        add: (self mutate: aChromosome2)]
                ifFalse: 
                    [population
                        add: (self clone: aChromosome1);
                        add: (self clone: aChromosome2)]]

\end{verbatim}
{\bf randomnizePopulation}
\begin{verbatim}
    self reset.
    [ self isFullyPopulated] whileFalse: [ population add: self 
                                                    randomChromosome].

\end{verbatim}
{\bf rateOfCrossover:} {\tt aNumber}
\begin{verbatim}
    (aNumber between: 0 and: 1) 
        ifFalse: [self error: 'Illegal rate of cross-over'].
    rateOfCrossover := aNumber

\end{verbatim}
{\bf rateOfMutation:} {\tt aNumber}
\begin{verbatim}
    (aNumber between: 0 and: 1) 
        ifFalse: [self error: 'Illegal rate of mutation'].
    rateOfMutation := aNumber

\end{verbatim}
{\bf reset}
\begin{verbatim}
    population := OrderedCollection new: populationSize.

\end{verbatim}


$$\halign{ #\hfil&\quad#\hfil\cr {\sl Class}& {\Large\bf DhbVectorChromosomeManager}\cr
{\sl Subclass of }&{\tt DhbChromosomeManager}\cr\noalign{\vskip 1ex}

{\sl Instance variable names:}&\parbox[t]{4 in}{\tt  origin range }\cr\noalign{\vskip 1ex}}$$


Instance methods
{\parskip 1ex\par\noindent}
{\bf crossover:} {\tt aChromosome1} {\bf and:} {\tt aChromosome2}
\begin{verbatim}
    | index new1 new2|
    index := ( aChromosome1 size - 1) random + 2.
    new1 := self clone: aChromosome1.
    new1 replaceFrom: index to: new1 size with: aChromosome2 
                                                    startingAt: index.
    new2 := self clone: aChromosome2.
    new2 replaceFrom: index to: new2 size with: aChromosome1 
                                                    startingAt: index.
    ^Array with: new1 with: new2

\end{verbatim}
{\bf mutate:} {\tt aVector}
\begin{verbatim}
    | index |
    index := aVector size random + 1.
    ^( aVector copy)
            at: index put: ( self randomComponent: index);
            yourself

\end{verbatim}
{\bf origin:} {\tt aVector}
\begin{verbatim}
    origin := aVector.

\end{verbatim}
{\bf randomChromosome}
\begin{verbatim}
    ^( ( 1 to: origin size) collect: [ :n | self randomComponent: n]) 
                                                              asVector

\end{verbatim}
{\bf randomComponent:} {\tt anInteger}
\begin{verbatim}
    ^( range at: anInteger) random + ( origin at: anInteger)

\end{verbatim}
{\bf range:} {\tt aVector}
\begin{verbatim}
    range := aVector.

\end{verbatim}


\end{listing}
Listing \ref{ls:optimizerabsgen} shows how the genetic optimizer
is implemented in Smalltalk. The following code example shows how
to use a genetic optimizer to find the maximum of a vector
function.
\begin{codeExample}
\begin{verbatim}

    | fBlock optimizer manager origin range result |
\end{verbatim}
 {\tt fBlock :=<\sl the goal function\tt >}\hfil\break
 {\tt origin :=<\sl a vector containing the minimum expected value of the component\tt >}\hfil\break
 {\tt range :=<\sl a vector containing the expected range of the component\tt >}\hfil\break
\begin{verbatim}
    optimizer := DhbGeneticOptimizer maximizingFunction: fBlock.
    manager := DhbVectorChromosomeManager new: 100 mutation: 0.1 crossover: 0.1.
    manager origin: origin; range: range.
    optimizer chromosomeManager: manager.
    result := optimizer evaluate.
\end{verbatim}
\end{codeExample}
After establishing the goal function and the search space, an
instance of the genetic optimizer is created. The next line
creates an instance of a vector chromosome manager for a
population of 100 individuals (sufficient for a 2-3 dimensional
space) and rates of mutation and crossover equal to $10\%$. The
next line defines the search space into the chromosome manager.
The final line performs the genetic search and returns the result.

In Smalltalk the population of the next generation is maintained
in the instance variable {\tt population}. Each time a next
generation has been established, it is transferred into a
collection of best points by the method {\tt collectPoints}. Each
element of the collection {\tt bestPoints} is an instance of an
subclass of {\tt OptimizingPoint}. The exact type of the class is
determined by the search strategy. Since best points are sorted
automatically, the result is always the position of the first
element of  {\tt bestPoints}.

\begin{listing} Smalltalk implementation of genetic algorithm \label{ls:optimizerabsgen}
$$\halign{ #\hfil&\quad#\hfil\cr {\sl Class}& {\Large\bf DhbGeneticOptimizer}\cr
{\sl Subclass of }&{\tt DhbFunctionOptimizer}\cr\noalign{\vskip 1ex}

{\sl Instance variable names:}&\parbox[t]{4 in}{\tt  chromosomeManager }\cr\noalign{\vskip 1ex}}$$


Class methods
{\parskip 1ex\par\noindent}
{\bf defaultMaximumIterations}
\begin{verbatim}
    ^500

\end{verbatim}
{\bf defaultPrecision}
\begin{verbatim}
    ^0

\end{verbatim}



Instance methods
{\parskip 1ex\par\noindent}
{\bf chromosomeManager:} {\tt aChromosomeManager}
\begin{verbatim}
    chromosomeManager := aChromosomeManager.
    ^self

\end{verbatim}
{\bf collectPoints}
\begin{verbatim}
    | bestPoint |
    bestPoints notEmpty
        ifTrue: [ bestPoint := bestPoints removeFirst].
    bestPoints removeAll: bestPoints asArray.
    chromosomeManager population do: [:each | self addPointAt: each].
    bestPoint notNil
        ifTrue: [ bestPoints add: bestPoint].
    result := bestPoints first position.

\end{verbatim}
{\bf computePrecision}
\begin{verbatim}
    ^1

\end{verbatim}
{\bf evaluateIteration}
\begin{verbatim}
    | randomScale |
    randomScale := self randomScale.
    chromosomeManager reset.
    [ chromosomeManager isFullyPopulated]
        whileFalse: [ self processRandomParents: randomScale].
    self collectPoints.
    ^self computePrecision

\end{verbatim}
{\bf initializeIterations}
\begin{verbatim}
    chromosomeManager randomnizePopulation.
    self collectPoints

\end{verbatim}
{\bf processRandomParents:} {\tt aNumberArray}
\begin{verbatim}
    chromosomeManager process: ( bestPoints at: ( self randomIndex: 
                                               aNumberArray)) position
                        and:  ( bestPoints at: ( self randomIndex: 
                                              aNumberArray)) position.

\end{verbatim}
{\bf randomIndex:} {\tt aNumberArray}
\begin{verbatim}
    | x n |
    x := Number random.
    n := 1.
    aNumberArray do: 
        [ :each |
          x < each
            ifTrue: [ ^n].
          n := n + 1.
        ].
    ^aNumberArray size  

\end{verbatim}
{\bf randomScale}
\begin{verbatim}
    | norm fBest fWorst answer|
    fBest := bestPoints first value.
    fWorst := bestPoints last value.
    norm := 1 / ( fBest - fWorst).
    answer := bestPoints collect: [ :each | (each value - fWorst) * 
                                                                norm].
    norm := 1 / ( answer inject: 0 into: [ :sum :each | each + sum]).
    fBest := 0.
    ^answer collect: [ :each | fBest := each * norm + fBest. fBest]

\end{verbatim}


\end{listing}

\subsection{Genetic algorithm --- Java abstract implementation}
\marginpar{Figure \ref{fig:joptimizingclasses} with the boxes {\bf
ChromosomeManager} and  {\bf GeneticOptimizer} grayed.} The Java
implementation of the genetic algorithm is done in two parts
because the type of the population is not known until one uses a
concrete chromosome manager. One could use casting in order to
preserve anonymity of the chromosome type at the level of the
genetic algorithm. I tend to avoid cast operator like the plague
and use them only when all other alternatives have failed. In this
case, casting is avoided by maintaining a concrete implementation
of the chromosome manager in pair with a class implementing the
genetic algorithm. This section describes the features of the
abstract classes. The code for these classes is shown in listings
\ref{lj:chromosome} and \ref{lj:optimizerabsgen}. The next section
describes a concrete implementation for vector functions.

The class {\tt ChromosomeManager} has an additional instance
variable --- called {\tt generator} --- used to keep an instance
of a random generator. Our implementation uses the default random
generator provided by Java. The method {\tt reset} allows the
creation of an empty next generation. The current size of the next
generation is obtained by calling method {\tt getPopulationSize}.
Other methods are provided to provide interaction with the genetic
algorithm class. The methods {\tt individualAt} returns the $k\th$
individual of the next generation, where $k$ is the supplied
integer argument. The methods {\tt addCloneOf}, {\tt
addMutationOf} add a new individual to the next generation by
applying respectively reproduction or mutation. The method {\tt
addCrossoversOf} add the two crossover offsprings of two
individuals. Here, the reader will see that casting becomes
unavoidable by the subclasses. However, since the objects taken as
arguments by the methods {\tt addCloneOf}, {\tt addMutationOf} and
{\tt addCrossoversOf} are coming from the method {\tt
individualAt} and only that method, the risk of error is minimal.

\begin{listing} Java abstract implementation of a chromosome
\label{lj:chromosome}
\input{Java/DhbOptimizing/ChromosomeManager.java}
\end{listing}

The class {\tt GeneticOptimizer} is an abstract class. The methods
providing the functionality to fill up the new population from the
next generation established by the chromosome manager are all
abstract methods. The method {\tt collectPoint} transfers a single
individual from the next generation to the population of mature
individuals.

A method {\tt initializeIterations} with one integer argument is
needed to allow the concrete class to initialize its memory for
the desired population size. This allows the instance variable
{\tt populationSize} to remain private.

\begin{listing} Java implementation of genetic algorithm
\label{lj:optimizerabsgen}
\input{Java/DhbOptimizing/GeneticOptimizer.java}
\end{listing}

\subsection{Genetic algorithm --- Java implementation with vectors}
\marginpar{Figure \ref{fig:joptimizingclasses} with the boxes {\bf
VectorChromosomeManager} and  {\bf VectorGeneticOptimizer}
grayed.} Listings \ref{lj:vectorchromosome} and
\ref{lj:optimizervector} show the concrete Java classes for the
chromosome manager and the genetic optimizer respectively. They
provide a concrete implementation of a genetic algorithm for
finding the optimum of a vector function. The following code
example shows how to find the maximum of a vector function using
these classes.
\begin{codeExample}
\begin{verbatim}
\end{verbatim}
{\tt ManyVariableFunction func = <\sl the goal function\tt
>}\hfil\break
 {\tt DhbVector origin =<\sl a vector containing the minimum expected value of the component\tt >}\hfil\break
 {\tt DhbVector range =<\sl a vector containing the expected range of the component\tt >}\hfil\break
\begin{verbatim}
    MaximizingPointFactory strategy = new MaximizingPointFactory();
    VectorChromosomeManager manager =
                            new VectorChromosomeManager( 100, 0.1, 0.1);
    manager.setOrigin( origin);
    manager.setRange( range);
    VectorGeneticOptimizer finder =
                    new VectorGeneticOptimizer( func, strategy, manager);
    finder.evaluate();
    double[] result = finder.getResult();
\end{verbatim}
\end{codeExample}
The line after the definition of the goal function and the
hypercube defining the search space creates an instance of a
maximizing vector. This will be the \patstyle{Strategy} of the
genetic optimizer. The next statement creates an instance of a
vector chromosome manager. Right after, two statements define the
search space. Then, the instance of the genetic optimizer is
created. The next statement performs the genetic algorithm and the
last statement retrieves the result.

The class {\tt VectorChromosomeManager} is a concrete subclass of
the class {\tt ChromosomeManager}. It maintains the next
generation in the instance variable {\tt population}. The instance
variable {\tt fillIndex} is used as an index when filling up the
next generation.

\begin{listing} Java implementation of a vector chromosome
\label{lj:vectorchromosome}
\input{Java/DhbOptimizing/VectorChromosomeManager.java}
\end{listing}
The class {\tt VectorGeneticOptimizer} is a concrete subclass of
the class {\tt GeneticOptimizer}. Our implementation chooses to
sort the points when they are collected. This is not really needed
as the algorithm for selecting the individual does not assumes
that the individuals are sorted by fitness. This is quite
practical, however, when following the behavior of the algorithm
with the debugger. For heavy duty usage the sorting ought to be
removed. If that is the case, the method {\tt getResult} must be
rewritten to fetch the fittest individual of the mature
population.
\begin{listing} Java implementation of genetic algorithm for vectors
\label{lj:optimizervector}
\input{Java/DhbOptimizing/VectorGeneticOptimizer.java}
\end{listing}

\section{Multiple strategy approach}
\label{sec:multistrategy} As we have seen most of the optimizing
algorithms described so far have some limitation:
\begin{itemize}
  \item Hill climbing algorithms may get into trouble far from the
  optimum and may get caught into a local optimum. This is exemplified in figure \ref{fig:hillvsrandom}.
  \item The simplex algorithm may get caught into a local optimum
  and does not converge well near the optimum.
  \item Genetic algorithms do not have a clear convergence
  criteria.
\end{itemize}
\begin{figure}
\center\psfig{figure=Figures/OptimizingComparisonvsd.eps,
width=12cm} \caption{Compared behavior of hill climbing and random
based algorithms.}\label{fig:hillvsrandom}
\end{figure}
After reading the above summary of the pro and cons of each
algorithm, the reader may have already come to the conclusion that
mixing the three algorithms together can make a very efficient
strategy to find the optimum of a wide variety of functions.

One can start with a genetic optimizer for a sufficient number of
iterations. This should ensure that the best points found at the
end of the search does not lie too far from the absolute optimum.
Then, one can use the simplex algorithm to get rapidly near the
optimum. The final location of the optimum is obtained using a
hill climbing optimizer.

\subsection{Multiple strategy approach --- General implementation}
This multiple strategy approach, inspired from the program MINUIT,
has been adapted to the use of the algorithms discussed here. The
class {\tt MultiVariableGeneralOptimizer} combines the three
algorithms: genetic, simplex and hill climbing, in this order. We
could have make it a subclass of {\tt Object}, but we decided to
reuse all the management provided by the abstract optimizer class
discussed in section \ref{sec:goptonedim}. Therefore, our general
purpose optimizer is a subclass of the abstract optimizer class
although it does not really uses the framework of an iterative
process. We only need one additional instance variable: the range
used to construct the hypercube search space for the vector
genetic chromosome manager. A corresponding setting method is
provided: {\tt setRange}.

The method {\tt initializeIterations} performs search using the
genetic algorithm as an option and, then, the simplex algorithm.
Since the genetic algorithm require a great deal of function
evaluate --- due to its stochastic nature --- it is a good idea to
give the user the choice of by-passing the use of the genetic
algorithm. If no range has been defined, only the simplex
algorithm is used from the supplied initial value. Otherwise a
search is made with the genetic algorithm using the initial value
and the range to define the search space. Then the simplex
algorithm is started from the best point found by the genetic
algorithm. The precision for the simplex search is set to the
square root of the precision for the final search. Less precision
is required for this step because the final search will give a
better precision.

The method {\tt evaluateIteration} performs the hill climbing
algorithm and returns it precision. As the desired precision of
the hill climbing algorithm is set to that of the general purpose
optimizer. As a consequence, there will only be a single
iteration.

Listing \ref{ls:optimizergeneral} shows the implementation in
Smalltalk. Listing \ref{lj:optimizergeneral} gives the code for
the Java implementation. At this point we shall abstain from
commenting the code as the reader should have no more need for
such thing$\ldots$ Hopefully!

\marginpar{Figure \ref{fig:soptimizingclasses} with the box {\bf
MultiVariableGeneralOptimizer} grayed.}
\begin{listing} Smalltalk
implementation of a general optimizer \label{ls:optimizergeneral}
$$\halign{ #\hfil&\quad#\hfil\cr {\sl Class}& {\Large\bf DhbMultiVariableGeneralOptimizer}\cr
{\sl Subclass of }&{\tt DhbFunctionOptimizer}\cr\noalign{\vskip 1ex}
}$$


Instance methods
{\parskip 1ex\par\noindent}
{\bf computeInitialValues}
\begin{verbatim}
    self range notNil
        ifTrue: [ self performGeneticOptimization].
    self performSimplexOptimization.
\end{verbatim}
{\bf evaluateIteration}
\begin{verbatim}
    | optimizer |
    optimizer := DhbHillClimbingOptimizer forOptimizer: self.
    optimizer desiredPrecision: desiredPrecision;
              maximumIterations: maximumIterations.
    result := optimizer evaluate.
    ^ optimizer precision
\end{verbatim}
{\bf origin}
\begin{verbatim}
    ^ result
\end{verbatim}
{\bf origin:} {\tt anArrayOrVector}
\begin{verbatim}
    result := anArrayOrVector.
\end{verbatim}
{\bf performGeneticOptimization}
\begin{verbatim}
    | optimizer manager |
    optimizer := DhbGeneticOptimizer forOptimizer: functionBlock.
    manager := DhbVectorChromosomeManager new: 100 mutation: 0.1 
                                                       crossover: 0.1.
    manager origin: self origin asVector; range: self range asVector.
    optimizer chromosomeManager: manager.
    result := optimizer evaluate.
\end{verbatim}
{\bf performSimplexOptimization}
\begin{verbatim}
    | optimizer manager |
    optimizer := DhbSimplexOptimizer forOptimizer: self.
    optimizer desiredPrecision: desiredPrecision sqrt;
              maximumIterations: maximumIterations;
              initialValue: result asVector.
    result := optimizer evaluate.
\end{verbatim}
{\bf range}
\begin{verbatim}
    ^ self bestPoints
\end{verbatim}
{\bf range:} {\tt anArrayOrVector}
\begin{verbatim}
    bestPoints := anArrayOrVector.
\end{verbatim}


\end{listing}

\marginpar{Figure \ref{fig:joptimizingclasses} with the box {\bf
MultiVariableGeneralOptimizer} grayed.}
\begin{listing} Java
implementation of a general optimizer \label{lj:optimizergeneral}
\input{Java/DhbOptimizing/MultiVariableGeneralOptimizer.java}
\end{listing}
\ifx\wholebook\relax\else\end{document}\fi
