\ifx\wholebook\relax\else
\documentclass[twoside]{book}
\usepackage[active]{srcltx}
\usepackage[LY1]{em}
\usepackage{epsfig}
\def\etc{{\it etc}}
\def\eg{{\it e.g.}}
\def\ie{{\it i.e.}}
\def\cf{{\it c.f.}\ }
\def\erf{\mathop{\rm erf}}
\def\sign{\mathop{\rm sign}}
\def\prob{\mathop{\rm Prob}}
\def\var{\mathop{\rm var}}
\def\mod{\mathop{\rm mod}}
\def\cor{\mathop{\rm cor}}
\def\cov{\mathop{\rm cov}}
\def\cl{\mathop{\rm CL}}
\def\kg{\mathop{\rm Kg}}
\def\patstyle#1{{\sc #1}}
\def\th{^{\mathop{\rm th}}}
\def\st#1{^{\mathop{\rm #1}}}
\def\note#1{\begin{quote}{\bf Note:} #1\end{quote}}
\def\braket#1{\left\langle #1\right\rangle}
\def\order#1{\let\o=#1{\cal O}\ifx\o 1$\left(n\right)$\else$\left(n^{#1}\right)$\fi}
\newtheorem{privListing}{Listing}[chapter]
\newenvironment{listing}{\vskip 3ex\hrule\vskip 1ex\begin{privListing}}{\end{privListing}\hrule\vskip 1ex}
\newtheorem{privExample}{Code example}[chapter]
\newenvironment{codeExample}{\begin{privExample}\begin{quote}\tt}{\end{quote}\end{privExample}}
\def\relboxl#1#2{\hbox to #1\hsize{#2\hfil}}
\def\relboxc#1#2{\hbox to #1\hsize{\hfil #2\hfil}}
\def\relboxr#1#2{\hbox to #1\hsize{\hfil #2}}
\def\transpose#1{{\bf #1}^{\mathop{\rm T}}}
\def\inverse#1{{\bf #1}^{-1}}
%\def\tm{$^{\mathop{\rm TM}}$}
\def\tm{ }
\newenvironment{mainEquation}{\marginpar[\vspace{3 ex} Main
equation$\Rightarrow$]{\vspace{3 ex}$\Leftarrow$Main
equation}\begin{equation}}{\end{equation}}
\def\rubrique#1{\paragraph{#1}\hfil\par\noindent}

\begin{document}
\fi

\chapter{Java primer for Smalltalk programmers}
\label{ch:javaprimer} This appendix is meant as a quick
introduction to the Java language for Smalltalkers. We shall give
minimum explanations to give the Smalltalker enough background to
be able to read the Java code presented in this book.

There are many books about Java on the market. My favorite book is
Java in a nutshell\cite{Java}. This book teaches Java assuming
that the reader is already familiar with C. A similar approach is
used here. This makes sense since the syntax of Java is very close
to that of C.

\section{Remarks on the syntax}
The designer of Java elected to use the existing syntax of C.
Compared to Smalltalk, this means a lot of redundant notation:
curly braces everywhere and a lot of reserved keywords. Statements
are terminated by a semicolon ({\tt ;}). We shall not explain the
syntax because it would be too long and we shall assume that most
Smalltalk programmers have already been exposed to C. Under this
assumption, we shall concentrate on the object-oriented features
of Java.

Like C, Java is file oriented.  An application is made of several
files loaded together. One file may contain several class
definitions. However, a class definition must be contained within
a single file. As a consequence, base classes as well as any class
obtained from a third party cannot be extended. The only
possibility is to extend them when it is possible\footnote{Java
has a qualifier {\tt final} which is a compiler optimization.
Classes with the final qualifier cannot have subclasses.}

\subsection{Classes}
A Java class is defined as follows:
\begin{verbatim}
 public class MyClass {}
\end{verbatim}
This defines a public class named {\tt MyClass} as a subclass of
the class {\tt Object} by default. Like in Smalltalk the
superclass of all classes is the class {\tt Object}. This
definition of {\tt MyClass} is not a very useful one since it has
no instance variable and no method. Instance variables and methods
are declared within the brackets following the class name. The
keyword {\tt public} states that this class is visible by any
other class.

\noindent The class {\tt MySubclass} is defined as a subclass of
{\tt MyClass} by the following declaration
\begin{verbatim}
 public class MySubclass  extends MyClass {}
\end{verbatim}

Instances of class are created via a {\sl constructor} method. The
constructor method has the same name as the class. The constructor
methods is called by supplying the keyword {\tt new} before the
class name. For example, here is how to create a new instance of
the class {\tt MyClass}:
\begin{verbatim}
 MyClass x = new MyClass();
\end{verbatim}
After this statement, the local variable {\tt x} points to the
newly created instance.

Constructors methods may have parameters. Since Java
differentiates methods by their names {\bf and} by the type of its
arguments, a class can have several constructor methods.

\subsection{Instance variables}
Instance variables must be declared with a type. An optional
qualifier defines the scope (or visibility) of the variable. An
instance variable is declared as follows
\begin{verbatim}
 protected double var1;
\end{verbatim}
This declares the variable {\tt var1} to be of type {\tt double},
a floating number with double precision. The keyword {\tt
protected} indicates that the variable is visible to all
subclasses of this class. This is the default when no qualifier is
supplied. Protected instance variables behaves like instance
variables in Smalltalk.

\subsection{Method declaration and method calling}
A Java method is declared with a type. An optional qualifier
defines the scope (or visibility) of the method. The name of the
method is followed by the list of parameters enclosed in
parentheses. The parentheses are always supplied, even if the list
of parameters is empty. Each parameter is defined by a type and an
identifier. This identifier can be used within the method code to
refer to the parameters. Like in Smalltalk, all parameters are
pointer to the objects. After the declaration the code of the
method follows enclosed within curly brackets.

\noindent Here is an example of a method declaration:
\begin{verbatim}
public DhbVector gradient( double x) {...}
\end{verbatim}
This method is declared as returning an object of type {\tt
DhbVector}. The argument of the method is a {\tt double}. The code
of the method must be located within curly brackets following the
method's declaration: our example does not show the code.

One interesting aspect of Java is that one can declare two methods
with the same name provided the type of the arguments is
different. For example, the following two methods can also be
defined within the same class where the above {\tt gradient}
method is defined:
\begin{verbatim}
 public DhbVector gradient( float x) {...}

 public DhbVector gradient( double x, double[] parameters) {...}
\end{verbatim}
At calling time, the Java virtual machine selects the proper
method depending of the type of the supplied argument(s).

The code of a Java method is of procedural nature. This is mainly
due to the use of the C syntax and constructs: {\tt for} and {\tt
while} loops, {\tt if} and {\tt if$\ldots$else}.

Calling a method is done by appending the name of the method to
the object to which the method is sent separated with a dot. For
example, the above methods can be called as follows:
\begin{verbatim}
 MyClass x = new MyClass();
 DhbVector v = x.gradient( Math.PI);
\end{verbatim}


\subsection{Objects and non-objects}
In Smalltalk everything is an object. This is not the case in
Java. There are so-called primitive types which are handled as
values like in conventional programming languages. The primitive
types are: {\tt char}, {\tt int}, {\tt long}, {\tt float}, {\tt
double} and {\tt boolean}.

In order to use values as objects, Java provides so-called
wrappers for each of the primitive type. For example, if a double
needs to be passed to a method expecting an argument of type {\tt
Object}, the double must be wrapped using the {\tt Double} class
wrapper as follows:
\begin{verbatim}
 double pi = 3.14159;
 myMethod( new Double(pi));
\end{verbatim}

\subsection{packages}
A Java package provide a solution to the name space, which has
been the holy grail of Smalltalk vendors for a long time. A Java
package provides a way to group cooperating classes together.
Classes from one package are not visible to classes of other
packages. Thus, two classes in an application may have the same
name as long as they are in different packages. To refer to both
of these classes in one common place, the designer must use the
package name followed by a dot and the class name. For example, if
you already have a class named {\tt Matrix} and that you want to
use the class {\tt Matrix} of this books, you must refer to the
class of this book as: {\tt DhbMatrixAlgebra.Matrix}.

The {\tt import} statement allows to make classes from another
package visible for all methods of a class. Then, a class can be
referred by its name only. Our Java code make ample use of import
statements.

\subsection{Scope qualifiers}
Java supports 3 types of scope qualifiers:
\begin{itemize}
  \item {\tt private}
  \item {\tt protected}
  \item {\tt public}
\end{itemize}
The meaning of the qualifiers is slightly different when applied
to classes, methods or instance variables.

A private instance variable is accessible only to the class where
this instance variable is declared. This means that a private
instance variable is {\sl not visible} to the subclass of a class
where the instance variable is declared. A protected instance
variable is accessible to all subclasses of the class where this
instance variable is declared. It is also visible to all classes
located in the same package as the class defining the variable. A
public instance variable is accessible by any object.

\noindent The scope qualifiers applied to a method have the same
properties than for instance variables.

A private class is accessible by all classes within the same file.
A protected class is visible to all classes located in the same
package. A public class is visible to any class.

The scope qualifiers are also used to optimize code execution.
This has an odd consequence for Smalltalkers: a private method of
a subclass called in a method from a superclass will not be
executed; the private method of the superclass is executed
instead. Let us make an example. Consider the two classes shown in
listing \ref{lj:inheritance}
\begin{listing} Inheritance of private methods \label{lj:inheritance}
\input{Java/InheritanceCheck/MyClass.java}
\input{Java/InheritanceCheck/MySubclass.java}
\end{listing}
The method {\tt privateCommon} executed within the method {\tt
privateSuper} is always that of the class {\tt MyClass}.
Smalltalkers should carefully study the code as it works against
their logic. Executing the main method of class {\tt MyClass}
yields the following result:
\begin{verbatim}
 Start running MyClass.privateCommon
    => Executing private method in superclass.
 Start running MySubclass.privateCommon
    => Executing private method in superclass.
 Start running MyClass.protectedCommon
    => Executing protected method in superclass.
 Start running MySubclass.protectedCommon
    => Executing protected method in subclass.
\end{verbatim}
Executing the main method of class {\tt MySubclass} yields the
following result:
\begin{verbatim}
 Start running MySubclass.privateSub
    => Executing private method in subclass.
\end{verbatim}
This surprising behavior is not a bug. It is a documented feature
of the Java compiler and virtual machine optimization. This
apparent breach of inheritance is a consequence of code
optimization.

\subsection{static qualifier}
The {\tt static} qualifier defines that a variable or a method is
attached to the class and not to an instance.

Smalltalkers can think of static variables as class variables.
Static methods can be considered as class variables. They are,
however, some subtle differences with the Smalltalk equivalents.

\section{Abstract class and interface}
In Java a class or a method can be declared as {\tt abstract}.

An abstract class is meant to be the start of a sub-hierarchy of
classes. In Smalltalk the concept of abstract class is purely a
matter of convention. In Java the abstract qualifier is enforced:
an abstract class cannot have any instances.

An abstract method can only be defined within an abstract class or
an interface (\cf below). It is defined as follows:
\begin{verbatim}
 protected abstract double computePrecision();
\end{verbatim}
Declaring a method as abstract instructs the Java compiler to
require that any concrete subclass must implement such a method: a
concrete subclass omitting to define concrete implementations of
the abstract methods of its superclasses is flagged as an error at
compiling time.

An interface is a new concept introduced by Java (at least to my
knowledge). One can consider an interface as an abstract class,
which cannot have any instance variable or concrete methods.
Interfaces come handy when constructing polymorphic behavior
between classes not part of the same hierarchy. Here is an example
of interface definition:
\begin{verbatim}
 public interface PlanarShape {
    protected abstract double getPerimeter();
    protected abstract double getArea();
 }
\end{verbatim}
Once an interface has been declared, it can be used as a type
declaration. For example the following method can be written in a
class having access to the interface defined just above:
\begin{verbatim}
 public double totalArea( PlanarShape[] p) {
    double totalArea = 0;
    for ( int i = 0; i < p.length; i++)
        totalArea += p.getArea();
    return totalArea;
 }
\end{verbatim}
At this point, the designer of the interface has no idea how the
method {\tt getArea} can be implemented.

\noindent A class declares itself as adhering to an interface as
follows:
\begin{verbatim}
 public class Square implements PlanarShape {
    private double side;
\end{verbatim}
$\ldots\ldots$
\begin{verbatim}
    double getArea() {
    return side * side;
    }
 }
\end{verbatim}
A class can adhere to several interfaces at the same time.

\section{Exception handling}
Exception handling is built in the language. By this I mean that
not only exception handling structures exist, like in Smalltalk,
but also that these structures are enforced by the compiler.

Exceptions are subclasses of the class {\tt Exception}. Like in
Smalltalk, a designer can define its own type of exception.
Exceptions have two constructor methods: one with no argument and
one with a string argument. The string argument is used to pass
additional information to the exception. This information is
displayed by the debugger or on the console when the exception is
not handled by the application.

In the Java jargon, an exception is {\sl thrown} by a method and
is {\sl caught} (or better {\tt catch}-ed) by a caller of that
method, not necessarily the direct caller. Here is an example of a
method throwing an exception:
\begin{verbatim}
 public double inverse( double x) throws MyZeroDivideException {
    if ( Math.abs( x) < DhbMath.getMachinePrecision() )
        throw new MyZeroDivideException();
    return 1 / x;
 }
\end{verbatim}
The declaration of the method uses the modifier {\tt throws
MyZeroDivideException} to signal the compiler that this method may
throw an exception of the specified class. The actual {\sl
throwing} of the exception is made by the statement {\tt throw}
followed by a new instance of the exception.

If a method is defined as throwing an exception, any method
calling the former must catch the exception. Otherwise, the
compiler issues an error. Catching the exception can be made
explicitly or implicitly.

Explicit exception catching is made with the statement {\tt
try$\ldots$catch}. Here is an example
\begin{verbatim}
 try{ double x = pivot();
      double inversePivot = inverse(x);
      } catch ( MyZeroDivideException e)
\end{verbatim}
\relboxr{1}{\tt\{<\sl code to handle the exception\tt
>\}}
\noindent To catch an exception, the statements where an exception
might occurs must be enclosed inside curly brackets preceded by
the key word {\tt try} and followed by the keyword {\tt catch}.
The keyword catch is followed by a parentheses enclosing a
declaration of the exception and a series of statements, enclosed
within curly brackets, which are handling the exception. The
variable corresponding to the exception --- {\tt e} in the example
above
--- is visible inside the block handling the exception.

Implicit exception catching is made by declaring that the method
is throwing the same exception. For example,
\begin{verbatim}
 public double inversePivot() throws MyZeroDivideException {
    double pivot = pivot();
    return inverse( pivot);
 }
\end{verbatim}
In this case the method {\tt inversePivot} delegates the handling
of the exception to one of its callers.

\section{Collections and related topics}
Smalltalkers, when first exposed to Java, surely are appalled at
the scarcity of the collection classes and at the absence of
decent iterator methods. Well, the sad thing is they are right!
The Java designer completely omitted this aspect of programming.

Arrays do exist. However, they are primitive types, not objects.
An array is declared with the keyword {\tt new} followed by the
type of the array elements and the dimension of the array enclosed
in square brackets. For example
\begin{verbatim}
    Cluster[] clusterArray = new Cluster[10];
\end{verbatim}
declares an array of 10 objects of the class {\tt Cluster}. The
reader should note that the type of an array is the type of the
element followed by {\tt []}. In our example the type of the
variable {\tt clusterArray} is {\tt Cluster[]}. All elements of an
array must be of the same type. Iterating over the elements of an
array must be done using index in a {\tt for} or {\tt while} loop.

To be complete, let us mention a few Java constructs which provide
more flexible collection behavior. They are not used in this book,
but I do not want to leave the Smalltalkers on a bad impression of
the Java language.

The Java class {\tt Vector} corresponds to the Smalltalk class
{\tt OrderedCollection}. The elements of a {\tt Vector} are of
type {\tt Object}. Its elements do not need to be of the same type
theoretically. In practice, however, handling a multi-type {\tt
Vector} is difficult as one needs to cast the element to the
proper type when extracting them from the {\tt Vector}.

\noindent The Smalltalk class {\tt Dictionary} corresponds to the
Java class {\tt HashTable}.

An equivalent to the Smalltalk {\tt do:} iterator method is
provided by the {\tt Enumeration} interface to which both {\tt
Vector} and {\tt HashTable} are adhering. Each element of the
collection is retrieved under the type {\tt Object}. However, this
forces the designer to cast each elements to its proper type
before using it within the iterator method.

\ifx\wholebook\relax\else\end{document}\fi
