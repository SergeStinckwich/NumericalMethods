\ifx\wholebook\relax\else
\documentclass[twoside]{book}
\usepackage[active]{srcltx}
\usepackage[LY1]{fontenc}
\usepackage{epsfig}
\def\etc{{\it etc}}
\def\eg{{\it e.g.}}
\def\ie{{\it i.e.}}
\def\cf{{\it c.f.}\ }
\def\erf{\mathop{\rm erf}}
\def\sign{\mathop{\rm sign}}
\def\prob{\mathop{\rm Prob}}
\def\var{\mathop{\rm var}}
\def\mod{\mathop{\rm mod}}
\def\cor{\mathop{\rm cor}}
\def\cov{\mathop{\rm cov}}
\def\cl{\mathop{\rm CL}}
\def\kg{\mathop{\rm Kg}}
\def\patstyle#1{{\sc #1}}
\def\th{^{\mathop{\rm th}}}
\def\st#1{^{\mathop{\rm #1}}}
\def\note#1{\begin{quote}{\bf Note:} #1\end{quote}}
\def\braket#1{\left\langle #1\right\rangle}
\def\order#1{\let\o=#1{\cal O}\ifx\o 1$\left(n\right)$\else$\left(n^{#1}\right)$\fi}
\newtheorem{privListing}{Listing}[chapter]
\newenvironment{listing}{\vskip 3ex\hrule\vskip 1ex\begin{privListing}}{\end{privListing}\hrule\vskip 1ex}
\newtheorem{privExample}{Code example}[chapter]
\newenvironment{codeExample}{\begin{privExample}\begin{quote}\tt}{\end{quote}\end{privExample}}
\def\relboxl#1#2{\hbox to #1\hsize{#2\hfil}}
\def\relboxc#1#2{\hbox to #1\hsize{\hfil #2\hfil}}
\def\relboxr#1#2{\hbox to #1\hsize{\hfil #2}}
\def\transpose#1{{\bf #1}^{\mathop{\rm T}}}
\def\inverse#1{{\bf #1}^{-1}}
%\def\tm{$^{\mathop{\rm TM}}$}
\def\tm{ }
\newenvironment{mainEquation}{\marginpar[\vspace{3 ex} Main
equation$\Rightarrow$]{\vspace{3 ex}$\Leftarrow$Main
equation}\begin{equation}}{\end{equation}}
\def\rubrique#1{\paragraph{#1}\hfil\par\noindent}

\begin{document}
\fi

\chapter{Accompanying CD ROM}
All the code presented in this book is available in the
accompanying CD ROM. The contents of the CD ROM is described in
the sections of this chapter.

Permission to use the code for any purpose is granted to the
purchaser of the book. The code is distributed on an "AS IS"
basis, without warranty. Neither the author nor Morgan Kaufmann,
Inc. make any representation, or warranty, either expressed or
implied, with respect to the code, its quality or accuracy, or
fitness for a specific purpose. Therefore, neither the author nor
Morgan Kaufmann, Inc. shall have any liability to you or any other
person or entity with respect to any liability, loss or damage
caused or alleged to have been caused directly or indirectly by
the code contained on the CD ROM. This includes, but is not
limited to, interruption of service, loss of data, loss of
classroom time, loss of consulting or anticipatory profits, or
consequential damages from the use of the code contained on the CD
ROM.

\section{Smalltalk code}
The Smalltalk code presented in this book was created with Visual
Age for Smalltalk under ENVY. The entire code is contained in a
single application. The application is split into several
subapplications: these subapplications correspond more or less to
the chapters of this book. To be honest, they correspond to the
chapters of the early version.
\subsection{non-ENVY users}
Readers using Smalltalk without ENVY must load the classes in
so-called {\it chunk} format. These files are located in the
directory {\tt Smalltlk} and all subdirectories. The names of the
subdirectories and files correspond to the subapplications of the
ENVY repository. These names have been modified to be used on
systems using an 8.3 file name convention.

The classes of early chapters are referenced by classes of
subsequent chapters. The reader should refer to figure
\ref{fig:roadmap} of chapter \ref{ch:introduction} for an explicit
description of the dependency between classes.

\noindent Here is a description of the files:
\begin{description}
  \item[\tt Smalltlk/DataMing/dataming.app] All classes defined in chapter \ref{ch:datamining}.
  \item[\tt Smalltlk/Estim/genestim.app] All classes defined in chapter \ref{ch:estimation} and
     extensions of classes {\tt DhbHistogram} and
     {\tt DhbStatisticalMoments} related to the $t$-test.
  \item[\tt Smalltlk/Estim/estimChi.app] Extensions of classes {\tt DhbHistogram} and
     {\tt DhbStatisticalMoments} related to the $\chi^2$-test.
  \item[\tt Smalltlk/Estim/estimF.app] Extensions of classes {\tt DhbHistogram} and
     {\tt DhbStatisticalMoments} related to the $F$-test.
  \item[\tt Smalltlk/Functs/general.app] The class {\tt DhbPolynomial} defined
     in chapter \ref{ch:function} and all classes defined in chapter \ref{ch:interpolation}.
  \item[\tt Smalltlk/Functs/beta.app] Extensions of class {\tt Number} defined
     in chapter \ref{ch:function} for the beta function.
  \item[\tt Smalltlk/Functs/errfunct.app] The class {\tt DhbErfApproximation} defined
     in chapter \ref{ch:function} and extensions of class {\tt Number} defined
     in chapter \ref{ch:function} for the error function.
  \item[\tt Smalltlk/Functs/gamma.app] The class {\tt DhbLanczosFormula} defined
     in chapter \ref{ch:function} and extensions of classes {\tt Number} and {\tt Integer} defined
     in chapter \ref{ch:function} for the gamma function.
  \item[\tt Smalltlk/Iterativ/iterativ.app] All classes defined in chapters \ref{ch:iteration},
     \ref{ch:zeroes}, \ref{ch:integration} and \ref{ch:series}.
  \item[\tt Smalltlk/LinAlgeb/linalgeb.app] The classes defined in chapter \ref{ch:linearalgebra},
   except for classes defined in the subapplications of this one.
  \item[\tt Smalltlk/LinAlgeb/eigenv.app] The classes {\tt DhbLargestEigenValueFinder}
  and {\tt DhbJacobiTransformation} defined in chapter \ref{ch:linearalgebra}.
  \item[\tt Smalltlk/LinAlgeb/determin.app] Extensions of classes {\tt DhbMatrix} and
     {\tt DhbLUPDecomposition} related to the computation of the matrix determinant.
  \item[\tt Smalltlk/LinAlgeb/inverse.app] Extensions of classes {\tt DhbMatrix}, {\tt DhbSymmetricMatrix} and
     {\tt DhbLUPDecomposition} related to the computation of matrix inversion.
  \item[\tt Smalltlk/Minimize/minimize.app] All classes defined in chapter \ref{ch:minimization}.
  \item[\tt Smalltlk/NumPrec/numprec.app] The class {\tt DhbFloatingPointMachine} defined in
  chapter \ref{ch:introduction} and the class {\tt DhbDecimalFloatingNumber} described in appendix \ref{ch-fpSimul}.
  \item[\tt Smalltlk/Statist/stats.app] All classes defined in chapter \ref{ch:statistics} as well as all
  classes describing a probability distribution (defined in chapters \ref{ch:statistics},
  \ref{ch:estimation} and appendix \ref{ch:distributions}).
\end{description}

\subsection{ENVY users}
Readers using ENVY\tm can load the ENVY\tm repository contained in
the file {\tt Vast/OONumS.dat}. This repository contains a single
application, {\tt DhbNumericalMethods}, containing the entire code
presented in this book. Here is a brief description of the
contents of each subapplication.
\begin{description}
  \item[\tt DhbDataMining] All classes defined in chapter \ref{ch:datamining}.
  \item[\tt DhbEstimation] All classes defined in chapter \ref{ch:estimation} and
     extensions of classes {\tt DhbHistogram} and
     {\tt DhbStatisticalMoments} related to the $t$ test.
  \item[\tt DhbEstimationChi2] Extensions of classes {\tt DhbHistogram} and
     {\tt DhbStatisticalMoments} related to the $\chi^2$ test.
  \item[\tt DhbEstimationF] Extensions of classes {\tt DhbHistogram} and
     {\tt DhbStatisticalMoments} related to the $F$ test.
  \item[\tt DhbFunctionEvaluation] The class {\tt DhbPolynomial} defined
     in chapter \ref{ch:function} and all classes defined in chapter \ref{ch:interpolation}.
  \item[\tt DhbBetaFunction] Extensions of class {\tt Number} defined
     in chapter \ref{ch:function} for the beta function.
  \item[\tt DhbErrorFunction] The class {\tt DhbErfApproximation} defined
     in chapter \ref{ch:function} and extensions of class {\tt Number} defined
     in chapter \ref{ch:function} for the error function.
  \item[\tt DhbGammaFunction] The class {\tt DhbLanczosFormula} defined
     in chapter \ref{ch:function} and extensions of classes {\tt Number} and {\tt Integer} defined
     in chapter \ref{ch:function} for the gamma function.
  \item[\tt DhbIterativeAlgorithms] All classes defined in chapters \ref{ch:iteration},
     \ref{ch:zeroes}, \ref{ch:integration} and \ref{ch:series}.
  \item[\tt DhbLinearAlgebra] The classes defined in chapter \ref{ch:linearalgebra},
   except for classes defined in the subapplications of this one.
  \item[\tt DhbEigenValuesAndVectors] The classes {\tt DhbLargestEigenValueFinder}
  and {\tt DhbJacobiTransformation} defined in chapter \ref{ch:linearalgebra}.
  \item[\tt DhbMatrixDeterminant] Extensions of classes {\tt DhbMatrix} and
     {\tt DhbLUPDecomposition} related to the computation of the matrix determinant.
  \item[\tt DhbMatrixInversion] Extensions of classes {\tt DhbMatrix}, {\tt DhbSymmetricMatrix} and
     {\tt DhbLUPDecomposition} related to the computation of matrix inversion.
  \item[\tt DhbMinimization] All classes defined in chapter \ref{ch:minimization}.
  \item[\tt DhbNumericalPrecision] The class {\tt DhbFloatingPointMachine} defined in
  chapter \ref{ch:introduction} and the class {\tt DhbDecimalFloatingNumber} described in appendix \ref{ch-fpSimul}.
  \item[\tt DhbStatistics] All classes defined in chapter \ref{ch:statistics} as well as all
  classes describing a probability distribution (defined in chapters \ref{ch:statistics},
  \ref{ch:estimation} and appendix \ref{ch:distributions}).
\end{description}

\section{Java code}
The Java code presented in this book was created with Visual Age
for Java. The entire code is split into several packages: these
packages correspond more or less to the chapters of this book.
They differ slightly from the Smalltalk applications because the
Java code was generated independently from the Smalltalk code.
Also due to the fact that Java does not allow to extend a class in
another file, all files contain only one class.

The following packages have been defined:
\begin{description}
  \item[\tt DhbDataMining] All classes defined in chapter \ref{ch:datamining}.
  \item[\tt DhbEstimation] All classes defined in chapter \ref{ch:estimation}.
  \item[\tt DhbFunctionEvaluation] All classes defined in
  chapters \ref{ch:introduction} and  \ref{ch:function}.
  \item[\tt DhbInterfaces] All interfaces defined in this book.
  \item[\tt DhbInterpolation] All classes defined in chapter \ref{ch:interpolation}.
  \item[\tt DhbIterations] All classes defined in chapters \ref{ch:iteration},
     \ref{ch:zeroes}, \ref{ch:integration} and \ref{ch:series}.
  \item[\tt DhbMatrixAlgebra] All classes defined in chapter \ref{ch:linearalgebra}.
  \item[\tt DhbOptimizing] All classes defined in chapter \ref{ch:minimization}.
  \item[\tt DhbScientificCurves] The classes {\tt Histogram} and
     {\tt Curve}.
  \item[\tt DhbStatistics] The classes defined in chapter \ref{ch:statistics}, except class {\tt Histogram}
  and the classes defined in appendix \ref{ch:distributions}.
\end{description}


\subsection{Visual Age for Java}
The file {\tt CDRom/Vaj/OONumJ.dat} contains a Visual Age
repository. Users of Visual for Java can load the entire code by
loading this file using the import facility of Visual Age.

\subsection{Other Java systems}
Classes must be loaded from the hierarchy of directories rooted at
{\tt CDRom/Java}. Each subdirectory corresponds to one of the
packages described earlier.

\note{Because some Java system requires that the name of the class
be identical to the name of the file, the original names of the
classes have been kept. Similarly, the name of the directory must
be equal to the name of the package. If you have transferred the
contents of the CD ROM over a system limiting the size of the file
name, you must first rename each file and directory before loading
the classes.}

\section{Utilities}
The directory {\tt Utility} can only be used on a 32-bit Windows
systems. It contains two programs to study some numerical effects
experimentally. The directory must be copied as is onto a 32-bit
Windows systems (Windows NT 4.0 or higher, Windows 98 or higher).
The executable files must be run from this directory because it
contains all the resources needed to execute them.

\subsection{Distribution demo}
The program contained in the file {\tt DistributionDemo.exe}
allows the reader to study all probability density functions
described in this book. After starting the program, a window
containing a note book will open. Each page of the notebook
corresponds to a distribution. The parameters can be modified.

Then, clicking on the button labeled {\tt Generate} will generate
random values distributed according to the distribution. A window
allows to modify the number of generated values and the parameters
of the histogram into which the values are accumulated. When the
generating is completed the resulting histogram is displayed. The
reader can then attempt a least square fit or a maximum likelihood
fit and view the result. The button labeled {\tt Integral} allows
to view the values as a distribution function.

\subsection{File reader}
This small utility program allows to display data read from a text
file, either in tab or coma delimited format, in graphical form.
Once data are read, the user can investigate interpolation, linear
regression and polynomial least square fit. This program was used
to generate the figures of chapters \ref{ch:interpolation} and
\ref{ch:estimation}.

After starting the program, a window containing the parameters of
the file containing the data will open. Here the reader can select
the name of the file to read, the format used (blank or coma
delimited) and which column contains which data. For error bars,
there are three cases:
\begin{enumerate}
  \item no error;
  \item the error is given explicitly;
  \item the error is computed from the standard deviation and the
  number of values used to compute the standard deviation (\cf
  equation \ref{eq:averagerror} in chapter \ref{ch:statistics}).
\end{enumerate}

If the data were read without error, the reader can investigate
the three interpolation algorithms described in chapter
\ref{ch:interpolation}. In addition linear regression and
polynomial least square fit can be tried, each point having an
identical weight.

If the data were read with errors, the reader can investigate only
Lagrange and Bulirsch-Stoer interpolation. Linear regression and
polynomial least square fit are performed using the supplied error
to compute the weight of each point as described in chapter
\ref{ch:estimation}.

\ifx\wholebook\relax\else\end{document}\fi
